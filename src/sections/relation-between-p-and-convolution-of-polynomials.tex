In this section we discuss the relation between $\polynomialP{m}{b}{x}$ and discrete convolution of
polynomials.
To show that $\polynomialP{m}{b}{x}$ involves the discrete convolution of polynomial $n^r$
let's remind the definition of $\polynomialP{m}{b}{x}$
\begin{equation*}
    \polynomialP{m}{b}{x} = \sum_{k=0}^{b-1} \sum_{r=0}^{m} \coeffA{m}{r} k^r (x-k)^r
    = \sum_{r=0}^{m} \coeffA{m}{r} \sum_{k=0}^{b-1} k^r (x-k)^r
\end{equation*}
A discrete convolution of defined over set of integers $\mathbb{Z}$ function $f$ is
\begin{equation*}
(f \ast f)[n]
    = \sum_{k} f(k) f(n-k)
\end{equation*}
General formula of discrete convolution for polynomials $f(n) = n^j, \; n\geq a \in \mathbb{R}$ may be derived immediately
\begin{equation*}
    \begin{split}
        \convPower{n}{j}{x}
        &=\sum_{k} k^j (x-k)^j [k\geq a][x-k\geq a] \\
        &=\sum_{k} k^j (x-k)^j [k\geq a][k\leq x-a] \\
        &=\sum_{k} k^j (x-k)^j [a \leq k \leq x-a] \\
        &=\sum_{k=a}^{x-a} k^j (x-k)^j,
    \end{split}
\end{equation*}
where $[a \leq k \leq x-a]$ is Iverson's bracket~\cite{APL}.
\begin{lem}
    \label{lemma_disc_conv_identity}
    For every $n\in\mathbb{N}, \; x\in\mathbb{R}$
    \[
        \convPower{n}{r}{x} = \sum_{k=0}^{x} k^r (x-k)^r, \quad n\geq 0.
    \]
\end{lem}
Thus, the corollary follows
\begin{cor}
    \label{cor_polynomial_p_and_macaulay_convolution}
    By Lemma~\ref{lemma_disc_conv_identity} the polynomial $\polynomialP{m}{b}{n}$ might be expressed in terms
    of discrete convolution as follows
    \[
        \polynomialP{m}{x+1}{x} = \sum_{r=0}^{m} \coeffA{m}{r} \convPower{n}{r}{x}, \quad n\geq 0.
    \]
\end{cor}
Therefore, another polynomial identity follows
\begin{thm}
    \label{thm_odd_power_by_macaulays_convolution}
    By Lemma~\ref{lemma_polynomial_p_and_odd_power}, Corollary~\ref{cor_polynomial_p_and_macaulay_convolution}
    and property~\ref{prop_p_identity}, for every $m\in\mathbb{N}, \; x\in\mathbb{R}$
    \begin{equation*}
        x^{2m+1} = -1 + \sum_{r=0}^{m} \coeffA{m}{r} \convPower{n}{r}{x}, \quad n\geq 0.
    \end{equation*}
\end{thm}
Now we notice the following identity in terms of polynomial $\polynomialP{m}{b}{x}$ and
discrete convolution $\convPower{n}{j}{x}$
\begin{prop}
    \label{prop_polynomial_p_and_macaulay_convolution_strict}
    For every $m \in \mathbb{N}, \; x\in\mathbb{R}$
    \begin{equation*}
        \begin{split}
            \polynomialP{m}{x}{x}
            &=\sum_{r=0}^{m} \coeffA{m}{r} \left(0^r x^r + \sum_{k=1}^{x-1} k^r (x-k)^r \right) \\
            &=\sum_{r=0}^{m} \coeffA{m}{r} 0^r x^r + \sum_{r=0}^{m} \coeffA{m}{r} \convPower{n}{r}{x} \\
            &=1 + \sum_{r=0}^{m} \coeffA{m}{r} \convPower{n}{r}{x}, \quad n\geq 1.
        \end{split}
    \end{equation*}
\end{prop}
Since that for all $r$ in $\coeffA{m}{r} 0^r x^r$ we have
\begin{equation*}
    \coeffA{m}{r} 0^r x^r =
    \begin{cases}
        1, & \mbox{if } r=0 \\
        0, & \mbox{if } r>0
    \end{cases}
\end{equation*}
Above is true because $\coeffA{m}{0}=1$ for every $m\in\mathbb{N}$, and $x^0 = 1$
for every $x$, ~\cite{Grah94SN}.
Hence, the following identity between $\polynomialP{m}{b}{x}$ and
discrete convolution $\convPower{n}{j}{x}$ holds
\begin{thm}
    \label{thm_odd_power_by_macaulays_convolution_strict}
    By Lemma~\ref{lemma_polynomial_p_and_odd_power} and
    Proposition~\ref{prop_polynomial_p_and_macaulay_convolution_strict},
    for every $m\in\mathbb{N}, \; x\in\mathbb{R}$
    \begin{equation*}
        x^{2m+1} = 1 + \sum_{r=0}^{m} \coeffA{m}{r} \convPower{n}{r}{x}, \quad n\geq 1.
    \end{equation*}
\end{thm}
\begin{cor}
    \label{cor_sum_of_coeffs_a}
    By Theorem~\ref{thm_odd_power_by_macaulays_convolution_strict}, for all $m\in\mathbb{N}$
    \begin{equation*}
        \sum_{r=0}^{m} \coeffA{m}{r} = 2^{2m+1} - 1
    \end{equation*}
\end{cor}
Corollary~\ref{cor_sum_of_coeffs_a} holds since that convolution $\convPower{n}{j}{x}=1, \; n\geq 1$
for each $r$ when $x=2$.