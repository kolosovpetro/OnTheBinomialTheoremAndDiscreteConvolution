We now set the following notation, which remains fixed for the remainder of this manuscript
\begin{itemize}
    \setlength\itemsep{1.6em}
    \item $\coeffA{m}{r}$ is a real coefficient defined recursively
    \begin{equation}
        \label{eq:def_coeff_a}
        \coeffA{m}{r} =
        \begin{cases}
        (2r+1)
            \binom{2r}{r} & \text{if } r=m \\
            (2r+1) \binom{2r}{r} \sum_{d=2r+1}^{m} \coeffA{m}{d} \binom{d}{2r+1} \frac{(-1)^{d-1}}{d-r}
            \bernoulli{2d-2r} & \text{if } 0 \leq r<m \\
            0 & \text{if } r<0 \text{ or } r>m
        \end{cases}
    \end{equation}
    where $m$ is non-negative integer and $\bernoulli{t}$ are Bernoulli numbers~\cite{WeissteinBernoulli}.
    It is assumed that $\bernoulli{1}=\frac{1}{2}$.

    \item $\polynomialP{m}{b}{x}$ is a $2m+1$-degree polynomial in $b,x\in\mathbb{R}$
    \begin{equation}
        \label{eq:def_polynomial_p}
        \polynomialP{m}{b}{x} = \sum_{k=0}^{b-1} \sum_{r=0}^{m} \coeffA{m}{r} k^r(x-k)^r
    \end{equation}

    \item $\coeffH{m}{t}{b}$ is a polynomial defined as
    \begin{equation}
        \label{eq:def_coeff_h}
        \coeffH{m}{t}{b}
        = \sum_{j=t}^{m} \binom{j}{t} \coeffA{m}{j} \frac{(-1)^j}{2j-t+1} \binom{2j-t+1}{b} \bernoulli{2j-t+1-b}
    \end{equation}
    integers $m,t,b$.

    \item $\polynomialX{m}{t}{j}$ is polynomial of degree $2m+1-t$ in $j\in\mathbb{R}$
    \begin{equation}
        \label{eq:def_coeff_x}
        \polynomialX{m}{t}{j} = (-1)^m \sum_{k=1}^{2m+1-t} \coeffH{m}{t}{k} \cdot j^k
    \end{equation}
    integers $m,t$.

    \item $\polynomialL{m}{x}{k}$ is $2m$ degree polynomial in $x,k\in\mathbb{R}$
    \begin{equation}
        \label{eq:def_polynomial_l}
        \polynomialL{m}{x}{k} = \sum_{r=0}^{m} \coeffA{m}{r} k^r(x-k)^r
    \end{equation}

    \item $(f\ast f)[n]$ is discrete convolution~\cite{damelin_discrete_convolution} of function $f$ defined over set of integers $\mathbb{Z}$
    \begin{align*}
    (f\ast f)[n]
        = \sum_{k} f(k) f(n-k)
    \end{align*}
    and its partial case for polynomials $n^j, \; n\geq a \in \mathbb{R}$
    \begin{align*}
        \convPower{n}{j}{x} =\sum_{k} k^j (x-k)^j [k\geq a][x-k\geq a] =\sum_{k=a}^{x-a} k^j (x-k)^j
    \end{align*}
\end{itemize}
