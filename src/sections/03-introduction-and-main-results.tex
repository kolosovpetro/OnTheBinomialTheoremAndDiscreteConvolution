The polynomial $\polynomialP{m}{b}{x}$ is a $2m+1$-degree polynomial in $x,b\in\mathbb{R}$ defined as
\begin{align*}
    \polynomialP{m}{b}{x} = \sum_{k=0}^{b-1} \sum_{r=0}^{m} \coeffA{m}{r} k^r(x-k)^r
\end{align*}
where $\coeffA{m}{r}$ is a real coefficient.
By means of Lemma~\eqref{lemma_polynomial_p_and_odd_power},
the polynomial $\polynomialP{m}{b}{x}$ has the following relation with Binomial theorem~\cite{AbraSteg72}
\begin{align*}
    \polynomialP{m}{x+y}{x+y} = \sum_{r=0}^{2m+1} \binom{2m+1}{r} x^{2m+1-r} y^r
\end{align*}
On the other hand, polynomial $\polynomialP{m}{b}{x}$ might be expressed in terms of discrete convolution
of polynomial $n^j$.
For every $n\geq 0$
\begin{align*}
    \polynomialP{m}{x+1}{x} = \sum_{r=0}^{m} \coeffA{m}{r} \convPower{n}{r}{x}
\end{align*}
It is important to notice that  $n^r$ of discrete convolution $\convPower{n}{r}{x}$ evaluated at $x$
is implicit piecewise-defined polynomial such as
\begin{equation*}
    n^{r} =
    \begin{cases}
        \underbrace{n \cdot n \cdots n}_{\mathrm{r \; times}}, & \mbox{if } n \geq 0 \\
        0, & \mbox{otherwise}
    \end{cases}
\end{equation*}
Therefore, it is easy to notice the following identities in terms of Binomial theorem and discrete convolution,
see the corollaries~\eqref{cor_bin_exp_and_macaulay_conv} and~\eqref{cor_bin_exp_and_macaulay_conv_strict}.
For every $n \geq 0$
\begin{equation*}
    \sum_{r=0}^{m} \coeffA{m}{r} \convPower{n}{r}{x+y}
    = 1 + \sum_{r=0}^{2m+1} \binom{2m+1}{r} x^{2m+1-r} y^r
\end{equation*}
For every $n > 0$
\begin{equation*}
    \sum_{r=0}^{m} \coeffA{m}{r} \convPower{n}{r}{x+y}
    = -1 + \sum_{r=0}^{2m+1} \binom{2m+1}{r} x^{2m+1-r} y^r
\end{equation*}
Additionally, the following generalizations for the multinomial case are discussed in
the corollaries~\eqref{cor_mult_exp_and_macaulay_conv} and ~\eqref{cor_mult_exp_and_macaulay_conv_strict}.
For every $n \geq 0$
\begin{align*}
    \sum_{r=0}^{m} \coeffA{m}{r} \convPower{n}{r}{\multifoldSum{t}} =
    1 + \sum_{\multifoldSum[k]{t}=2m+1} \binom{2m+1}{k_1, k_2,\ldots, k_t} \prod_{\ell=1}^{t} x_\ell^{k_\ell}
\end{align*}
For every $n>0$
\begin{align*}
    \sum_{r=0}^{m} \coeffA{m}{r} \convPower{n}{r}{\multifoldSum{t}} =
    -1 + \sum_{\multifoldSum[k]{t}=2m+1} \binom{2m+1}{k_1, k_2,\ldots, k_t} \prod_{\ell=1}^{t} x_\ell^{k_\ell}
\end{align*}
A few polynomial identities are straightforward by means of
the theorems~\eqref{thm_odd_power_by_macaulays_convolution},~\eqref{thm_odd_power_by_macaulays_convolution_strict}.
More precisely, by the theorem~\eqref{thm_odd_power_by_macaulays_convolution} we have an odd-power identity as follows
\begin{equation*}
    x^{2m+1} = \sum_{r=0}^{m} \coeffA{m}{r} \sum_{k=0}^{x-1} k^r (x-k)^r
\end{equation*}
so that
\begin{align*}
    1 + x^{2m+1} = \sum_{r=0}^{m} \coeffA{m}{r} \convPower{n}{r}{x}
    = \sum_{r=0}^{m} \coeffA{m}{r} \sum_{k=0}^{x} k^r (x-k)^r
\end{align*}
From the other side, the theorem~\eqref{thm_odd_power_by_macaulays_convolution_strict} provides an odd-power
polynomial identity as follows
\begin{equation*}
    x^{2m+1} = \sum_{r=0}^{m} \coeffA{m}{r} \sum_{k=1}^{x} k^r (x-k)^r
\end{equation*}
so that
\begin{equation*}
    -1 + x^{2m+1} = \sum_{r=0}^{m} \coeffA{m}{r} \convPower{n}{r}{x}
    = \sum_{r=0}^{m} \coeffA{m}{r} \sum_{k=1}^{x-1} k^r (x-k)^r
\end{equation*}
For example,
\begin{align*}
    x^3 &= \sum_{k=1}^{x} 6k (x-k) + 1 \\
    x^5 &= \sum_{k=1}^{x} 30k^2 (x-k)^2 + 1 \\
    x^7 &= \sum_{k=1}^{x} 140 k^3 (x-k)^3 - 14k(x-k) + 1 \\
    x^9 &= \sum_{k=1}^{x} 630 k^4(x-k)^4 - 120k(x-k) + 1 \\
    x^{11} &= \sum_{k=1}^{x} 2772 k^5 (x-k)^5 + 660 k^2(x-k)^2 - 1386k(x-k) + 1 \\
    x^{13} &= \sum_{k=1}^{x} 51480 k^7 (x-k)^7 - 60060 k^3 (x-k)^3 + 491400 k^2 (x-k)^{2} - 450054 k (x-k) + 1 \\
\end{align*}
