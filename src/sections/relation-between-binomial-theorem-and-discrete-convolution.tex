\begin{cor}
    \label{cor_bin_exp_and_macaulay_conv}
    (Generalization of Theorem~\ref{thm_odd_power_by_macaulays_convolution} for Binomials.)
    For every $m\in\mathbb{N}, \; x,y\in\mathbb{R}$
    \begin{equation*}
        \sum_{r=0}^{m} \coeffA{m}{r} \convPower{n}{r}{x+y}
        =
        1 + \sum_{r=0}^{2m+1} \binom{2m+1}{r} x^{2m+1-r} y^r, \quad n\geq 0.
    \end{equation*}
\end{cor}
For example, given $m=0,1,2$ the Corollary~\ref{cor_bin_exp_and_macaulay_conv} yields
\begin{equation*}
    \begin{split}
        \sum_{r=0}^{0} \coeffA{0}{r} \convPower{n}{r}{x+y}
        &= 1 + x + y \\
        \sum_{r=0}^{1} \coeffA{1}{r} \convPower{n}{r}{x+y}
        &= 1 + x + y - (x + y) (1 + x + y) (1 - 3 x - 3 y + 2 (x + y)) \\
        &= 1 + x^3 + 3 x^2 y + 3 x y^2 + y^3\\
        \sum_{r=0}^{2} \coeffA{2}{r} \convPower{n}{r}{x+y}
        &=1 + x + y + (x + y) (1 + x + y) \left(-1 + x + 5 x^2 + y + 10 x y + 5 y^2\right. \\
        &-15 x (x + y) + 10 x^2 (x + y) - 15 y (x + y) + 20 x y (x + y) \\
        &+ 10 y^2 (x + y) +9 (x + y)^2 - 15 x (x + y)^2 \\
        &\left.-15 y (x + y)^{2} + 6 {(x + y)}^{3}\right) \\
        &=x^5 + 5 x^4 y + 10 x^3 y^2 + 10 x^2 y^3 + 5 x y^4 + y^5 + 1
    \end{split}
\end{equation*}
Above example could be verified using using the commands defined in Mathematica package at~\cite{PK22Source}
\begin{itemize}
    \item \texttt{BinomialTheoremAndDiscreteConvolutionTest[0, x + y]}
    \item \texttt{BinomialTheoremAndDiscreteConvolutionTest[1, x + y]}
    \item \texttt{Expand[BinomialTheoremAndDiscreteConvolutionTest[1, x + y]]}
    \item \texttt{BinomialTheoremAndDiscreteConvolutionTest[2, x + y]}
    \item \texttt{Expand[BinomialTheoremAndDiscreteConvolutionTest[2, x + y]]}
\end{itemize}
\begin{cor}
    \label{cor_bin_exp_and_macaulay_conv_strict}
    (Generalization of Theorem~\ref{thm_odd_power_by_macaulays_convolution_strict} for Binomials.)
    For every $m\in\mathbb{N}, \; x,y\in\mathbb{R}$
    \begin{equation*}
        \sum_{r=0}^{m} \coeffA{m}{r} \convPower{n}{r}{x+y}
        =
        -1 + \sum_{r=0}^{2m+1} \binom{2m+1}{r} x^{2m+1-r} y^r, \quad n > 0.
    \end{equation*}
\end{cor}
For example, given $m=0,1$ the Corollary~\ref{cor_bin_exp_and_macaulay_conv_strict} gives
\begin{equation*}
    \begin{split}
        \sum_{r=0}^{0} \coeffA{0}{r} \convPower{n}{r}{x+y}
        &= x + y - 1 \\
        \sum_{r=0}^{1} \coeffA{1}{r} \convPower{n}{r}{x+y}
        &= -1 + x + y - (-1 + x + y) (x + y) (-1 - 3 x - 3 y + 2 (x + y)) \\
        &= x^3 + 3 x^2 y + 3 x y^2 + y^3 - 1
    \end{split}
\end{equation*}
Above example could be verified using using the commands defined in Mathematica package at~\cite{PK22Source}
\begin{itemize}
    \item \texttt{BinomialTheoremAndDiscreteConvolutionStrictTest[0, x + y]}
    \item \texttt{BinomialTheoremAndDiscreteConvolutionStrictTest[1, x + y]}
    \item \texttt{Expand[BinomialTheoremAndDiscreteConvolutionStrictTest[1, x + y]]}
\end{itemize}
From the other prospective, let be a function $f_r(t,k) = (t-k)^r, \; t \geq k$, then following identity holds
\begin{equation}
(x-2a)
    ^{2m+1} + 1 =\sum_{r=0}^{m} \coeffA{m}{r} (f_r(t,k) \ast f_r(t,k))[x]
    \label{eq:parametric-identity}
\end{equation}
Let be a function $g_r(t,k) = (t-k)^r, \; t > k$, then
\begin{equation}
(x-2a)
    ^{2m+1} - 1 =\sum_{r=0}^{m} \coeffA{m}{r} (g_r(t,k) \ast g_r(t,k))[x]
    \label{eq:parametric-identity-strict}
\end{equation}

\subsection{Generalization for Multinomials} \label{subsec:generalization-for-multinomials}
In this subsection we generalize
Theorems~\eqref{thm_odd_power_by_macaulays_convolution} and~\eqref{thm_odd_power_by_macaulays_convolution_strict}
for multinomial cases.
\begin{cor}
    \label{cor_mult_exp_and_macaulay_conv}
    (Generalization of Theorem~\ref{thm_odd_power_by_macaulays_convolution} for Multinomials.)
    For every $x_1, x_2, \ldots, x_t\in\mathbb{R}, \; m\in\mathbb{N}, \; n\geq1\in\mathbb{N}$
    \[
        \sum_{r=0}^{m} \coeffA{m}{r} \convPower{n}{r}{\multifoldSum{t}} =
        1 + \sum_{\multifoldSum[k]{t}=2m+1} \binom{2m+1}{k_1, k_2,\ldots, k_t} \prod_{\ell=1}^{t} x_\ell^{k_\ell}
    \]
\end{cor}
For instance, given $m=1$ the Corollary~\ref{cor_mult_exp_and_macaulay_conv} gives
\begin{equation*}
    \begin{split}
        &\sum_{r=0}^{1} \coeffA{1}{r} \convPower{n}{r}{x+y+z} \\
        &=1 + x + y + z - (x + y + z) (1 + x + y + z) (1 - 3 x - 3 y - 3 z + 2 (x + y + z)) \\
        &=1 + x^3 + 3 x^2 y + 3 x y^2 + y^3 + 3 x^2 z + 6 x y z + 3 y^2 z + 3 x z^2 + 3 y z^2 + z^3.
    \end{split}
\end{equation*}
Above example could be verified using using the commands defined in Mathematica package at~\cite{PK22Source}
\begin{itemize}
    \item \texttt{BinomialTheoremAndDiscreteConvolutionTest[1, x + y + z]}
    \item \texttt{Expand[BinomialTheoremAndDiscreteConvolutionTest[1, x + y + z]]}
\end{itemize}
\begin{cor}
    \label{cor_mult_exp_and_macaulay_conv_strict}
    (Generalization of Theorem~\ref{thm_odd_power_by_macaulays_convolution_strict} for Multinomials.)
    For each $\multifoldSum{t} \geq 1, \; x_1,x_2,\ldots,x_t\in\mathbb{R}, \; m\in\mathbb{N}, \; n\geq1\in\mathbb{N}$
    \[
        \sum_{r=0}^{m} \coeffA{m}{r} \convPower{n}{r}{\multifoldSum{t}} =
        -1 + \sum_{\multifoldSum[k]{t}=2m+1} \binom{2m+1}{k_1, k_2,\ldots, k_t} \prod_{\ell=1}^{t} x_\ell^{k_\ell}
    \]
\end{cor}
For example, given $m=1$ the Corollary~\ref{cor_mult_exp_and_macaulay_conv_strict} gives
\begin{equation*}
    \begin{split}
        &\sum_{r=0}^{1} \coeffA{1}{r} \convPower{n}{r}{x+y+z} \\
        &=-1 + x + y + z - (-1 + x + y + z) (x + y + z) (-1 - 3 x - 3 y - 3 z + 2 (x + y + z)) \\
        &=-1 + x^3 + 3 x^2 y + 3 x y^2 + y^3 + 3 x^2 z + 6 x y z + 3 y^2 z + 3 x z^2 + 3 y z^2 + z^3.
    \end{split}
\end{equation*}
Above example could be verified using using the commands defined in Mathematica package at~\cite{PK22Source}
\begin{itemize}
    \item \texttt{BinomialTheoremAndDiscreteConvolutionStrictTest[1, x + y + z]}
    \item \texttt{Expand[BinomialTheoremAndDiscreteConvolutionStrictTest[1, x + y + z]]}
\end{itemize}
