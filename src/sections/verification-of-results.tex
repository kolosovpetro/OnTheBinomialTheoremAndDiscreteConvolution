To fulfill our study we provide an opportunity to verify its results by means of Wolfram Mathematica language.

\subsection{Mathematica commands} \label{subsec:mathematica-commands}
Proceeding to the repository~\cite{PK22Source} reader is able to find there a folder named \texttt{mathematica}
that contains the files
\begin{itemize}
    \item \texttt{OnTheBinomialTheoremAndDiscreteConvolution.m} is a package file with definitions
    \item \texttt{OnTheBinomialTheoremAndDiscreteConvolution.nb} is a notebook file with examples.
\end{itemize}
The following commands may be used to reproduce the results of this manuscript:
\begin{itemize}
    \item \texttt{A[m, r]} returns the real coefficient $\coeffA{m}{r}$ defined by~\eqref{eq:def_coeff_a}.
    \item \texttt{PrintTriangleOfA[rows]} prints the table of coefficients $\coeffA{m}{r}$. \\
    Command \texttt{PrintTriangleOfA[7]} reproduces the table (\ref{tab:table_of_coefficients_a}).
    \item \texttt{PolynomialL[m, n, k]} returns the polynomial $\polynomialL{m}{n}{k}$ defined by~\eqref{eq:def_polynomial_l}.
    \item \texttt{PolynomialP[m, x, b]} returns the polynomial $\polynomialP{m}{b}{x}$ defined by~\eqref{eq:def_polynomial_p}.
    \item \texttt{Expand[PolynomialP[m, x + y, x + y]]} verifies the Lemma~\ref{lemma_polynomial_p_and_odd_power}.
    \item \texttt{PolynomialH[m, t, j]} returns the polynomial $\coeffH{m}{t}{j}$ defined by~\eqref{eq:def_coeff_h}.
    \item \texttt{PolynomialX[m, t, k]} returns the polynomial $\polynomialX{m}{t}{k}$ defined by~\eqref{eq:def_coeff_x}.
    \item \texttt{Expand[BinomialTheoremAndDiscreteConvolutionTest[m, x + y]]} verifies the Corollary~\ref{cor_bin_exp_and_macaulay_conv}.
    \item \texttt{Expand[BinomialTheoremAndDiscreteConvolutionStrictTest[m, x + y]]} verifies the Corollary~\ref{cor_bin_exp_and_macaulay_conv_strict}.
    \item \texttt{DiscreteConvolutionPowerIdentityParametricTest[m, x, a]} verifies an equation~\eqref{eq:parametric-identity}.
    Usage \texttt{Column[Table[DiscreteConvolutionPowerIdentityParametricTest[1, x, 1], {x, 3, 20}], Left]}.
    \item \texttt{DiscreteConvolutionPowerIdentityStrictParametricTest[m, x, a]} verifies an equation~\eqref{eq:parametric-identity-strict}.
    Usage \texttt{Column[Table[DiscreteConvolutionPowerIdentityStrictParametricTest[1, x, 1], {x, 3, 20}], Left]}.
    \item \texttt{PolynomialIdentityInvolvingX[m, x, b]} validates an identity at~\eqref{eq:p_all_forms}
    \[\polynomialP{m}{b}{x} = \sum_{r=0}^{m} (-1)^{m-r} \polynomialX{m}{r}{b} \cdot x^r\]
    \item \texttt{PolynomialIdentityInvolvingH[m, n, b]} validates an identity at~\eqref{eq:p_all_forms}.
    \[\polynomialP{m}{b}{x} =\sum_{r=0}^{m} \sum_{\ell=1}^{2m-r+1} (-1)^{2m-r} \coeffH{m}{r}{\ell} \cdot b^\ell \cdot x^r\]
\end{itemize}

\subsection{Examples} \label{subsec:examples}
For example, given $m=1$ we have the following values of $\polynomialL{1}{x}{k}$
\begin{table}[H]
    \begin{tabular}{c|cccccccc}
        $x/k$ & 0 & 1  & 2  & 3  & 4  & 5  & 6  & 7 \\[3px]
        \hline
        0     & 1 &    &    &    &    &    &    &   \\
        1     & 1 & 1  &    &    &    &    &    &   \\
        2     & 1 & 7  & 1  &    &    &    &    &   \\
        3     & 1 & 13 & 13 & 1  &    &    &    &   \\
        4     & 1 & 19 & 25 & 19 & 1  &    &    &   \\
        5     & 1 & 25 & 37 & 37 & 25 & 1  &    &   \\
        6     & 1 & 31 & 49 & 55 & 49 & 31 & 1  &   \\
        7     & 1 & 37 & 61 & 73 & 73 & 61 & 37 & 1
    \end{tabular}
    \caption{Values of $\polynomialL{1}{x}{k}$. See OEIS entry: \href{https://oeis.org/A300656}{\texttt{A300656}}.}
    \label{tab:tab_3}
\end{table}
Table~\ref{tab:tab_3} can be reproduced using Mathematica command
\begin{center}
    \texttt{PrintTriangleOfPolynomialL[1, 7]}
\end{center}
defined in the~\cite{PK22Source}.
From Table~\ref{tab:tab_3} it is seen that
\begin{equation*}
    \begin{split}
        \polynomialP{1}{0}{0} &= 0 = 0^3 \\
        \polynomialP{1}{1}{1} &= 1 = 1^3 \\
        \polynomialP{1}{2}{2} &= 1+7 = 2^3 \\
        \polynomialP{1}{3}{3} &= 1+13+13 = 3^3 \\
        \polynomialP{1}{4}{4} &= 1+19+25+19 = 4^3 \\
        \polynomialP{1}{5}{5} &= 1+25+37+37+25 = 5^3
    \end{split}
\end{equation*}
Another case, given $m=2$ we have the following values of $\polynomialL{2}{x}{k}$
\begin{table}[H]
    \begin{tabular}{c|cccccccc}
        $x/k$ & 0 & 1    & 2    & 3    & 4    & 5    & 6    & 7 \\ [3px]
        \hline
        0     & 1 &      &      &      &      &      &      &   \\
        1     & 1 & 1    &      &      &      &      &      &   \\
        2     & 1 & 31   & 1    &      &      &      &      &   \\
        3     & 1 & 121  & 121  & 1    &      &      &      &   \\
        4     & 1 & 271  & 481  & 271  & 1    &      &      &   \\
        5     & 1 & 481  & 1081 & 1081 & 481  & 1    &      &   \\
        6     & 1 & 751  & 1921 & 2431 & 1921 & 751  & 1    &   \\
        7     & 1 & 1081 & 3001 & 4321 & 4321 & 3001 & 1081 & 1
    \end{tabular}
    \caption{Values of $\polynomialL{2}{x}{k}$. See OEIS entry: \href{https://oeis.org/A300656}{\texttt{A300656}}.}
    \label{tab:tab_4}
\end{table}
Table~\ref{tab:tab_4} can be reproduced using Mathematica command
\begin{center}
    \texttt{PrintTriangleOfPolynomialL[2, 7]}
\end{center}
defined in the~\cite{PK22Source}.
Again, an odd-power identity~\ref{lemma_polynomial_p_and_odd_power} holds
\begin{equation*}
    \begin{split}
        \polynomialP{2}{0}{0} &= 0 = 0^5 \\
        \polynomialP{2}{1}{1} &= 1 = 1^5 \\
        \polynomialP{2}{2}{2} &= 1+31 = 2^5 \\
        \polynomialP{2}{3}{3} &= 1+121+121 = 3^5 \\
        \polynomialP{2}{4}{4} &= 1+271+481+271 = 4^5 \\
        \polynomialP{2}{5}{5} &= 1+481+1081+1081+481 = 5^5
    \end{split}
\end{equation*}
