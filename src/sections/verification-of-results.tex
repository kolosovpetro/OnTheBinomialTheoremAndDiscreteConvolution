To fulfill our study we provide an opportunity to verify its results by means of Wolfram Mathematica language.

\subsection{Mathematica commands} \label{subsec:mathematica-commands}
Proceeding to the repository~\cite{mmca_package} the following commands to verify the formulas:
\begin{itemize}
    \item \texttt{A[m,r]} gives $\coeffA{m}{r}$, definition~\eqref{eq:def_coeff_a}.
    \item \texttt{L[m,n,k]} gives $\polynomialL{m}{n}{k}$, definition~\eqref{eq:def_polynomial_l}.
    \item \texttt{P[m,x,b]} gives $\polynomialP{m}{x}{b}$, definition~\eqref{eq:def_polynomial_p}.
    \item \texttt{P[m,x+y,x+y]} verifies Lemma~\ref{lemma_polynomial_p_and_odd_power}.
    \item \texttt{H[m,t,j]} gives $\coeffH{m}{t}{j}$.
    \item \texttt{X[m,t,k]} gives $\polynomialX{m}{t}{k}$.
    \item \texttt{ConvPowerIdentityStrict[m, x+y]} verifies the Corollary~\ref{cor_bin_exp_and_macaulay_conv_strict}.
    \item \texttt{ConvPowerIdentity[m, x+y]} verifies the Corollary~\ref{cor_bin_exp_and_macaulay_conv}.
    \item \texttt{ConvPowerIdentityParametric[m, x, a]}, $m,x,a$ are constants - verifies equation~\eqref{eq:parametric-identity}.
    \item \texttt{ConvPowerIdentityStrictParametric[m, x, a]}, $m,x,a$ are constants - verifies equation~\eqref{eq:parametric-identity-strict}.
\end{itemize}

\subsection{Examples} \label{subsec:examples}
For example, given $m=1$ we have the following values of $\polynomialL{1}{x}{k}$
\begin{table}[H]
    \begin{tabular}{c|cccccccc}
        $x/k$ & 0 & 1  & 2  & 3  & 4  & 5  & 6  & 7 \\[3px]
        \hline
        0     & 1 &    &    &    &    &    &    &   \\
        1     & 1 & 1  &    &    &    &    &    &   \\
        2     & 1 & 7  & 1  &    &    &    &    &   \\
        3     & 1 & 13 & 13 & 1  &    &    &    &   \\
        4     & 1 & 19 & 25 & 19 & 1  &    &    &   \\
        5     & 1 & 25 & 37 & 37 & 25 & 1  &    &   \\
        6     & 1 & 31 & 49 & 55 & 49 & 31 & 1  &   \\
        7     & 1 & 37 & 61 & 73 & 73 & 61 & 37 & 1
    \end{tabular}
    \caption{Values of $\polynomialL{1}{x}{k}$.}
    \label{tab:tab_3}
\end{table}
From Table~\ref{tab:tab_3} it is seen that
\begin{equation*}
    \begin{split}
        \polynomialP{1}{0}{0} &= 0 = 0^3 \\
        \polynomialP{1}{1}{1} &= 1 = 1^3 \\
        \polynomialP{1}{2}{2} &= 1+7 = 2^3 \\
        \polynomialP{1}{3}{3} &= 1+13+13 = 3^3 \\
        \polynomialP{1}{4}{4} &= 1+19+25+19 = 4^3 \\
        \polynomialP{1}{5}{5} &= 1+25+37+37+25 = 5^3
    \end{split}
\end{equation*}
Another case, given $m=2$ we have the following values of $\polynomialL{2}{x}{k}$
\begin{table}[H]
    \begin{tabular}{c|cccccccc}
        $x/k$ & 0 & 1    & 2    & 3    & 4    & 5    & 6    & 7 \\ [3px]
        \hline
        0     & 1 &      &      &      &      &      &      &   \\
        1     & 1 & 1    &      &      &      &      &      &   \\
        2     & 1 & 31   & 1    &      &      &      &      &   \\
        3     & 1 & 121  & 121  & 1    &      &      &      &   \\
        4     & 1 & 271  & 481  & 271  & 1    &      &      &   \\
        5     & 1 & 481  & 1081 & 1081 & 481  & 1    &      &   \\
        6     & 1 & 751  & 1921 & 2431 & 1921 & 751  & 1    &   \\
        7     & 1 & 1081 & 3001 & 4321 & 4321 & 3001 & 1081 & 1
    \end{tabular}
    \caption{Values of $\polynomialL{2}{x}{k}$.}
    \label{tab:tab_4}
\end{table}
Again, an odd-power identity~\ref{lemma_polynomial_p_and_odd_power} holds
\begin{equation*}
    \begin{split}
        \polynomialP{2}{0}{0} &= 0 = 0^5 \\
        \polynomialP{2}{1}{1} &= 1 = 1^5 \\
        \polynomialP{2}{2}{2} &= 1+31 = 2^5 \\
        \polynomialP{2}{3}{3} &= 1+121+121 = 3^5 \\
        \polynomialP{2}{4}{4} &= 1+271+481+271 = 4^5 \\
        \polynomialP{2}{5}{5} &= 1+481+1081+1081+481 = 5^5
    \end{split}
\end{equation*}
Tables ~\ref{tab:tab_3}, ~\ref{tab:tab_4} are entries \href{https://oeis.org/A287326}{\texttt{A287326}},
\href{https://oeis.org/A300656}{\texttt{A300656}} in~\cite{Sloane_theencyclopedia}.