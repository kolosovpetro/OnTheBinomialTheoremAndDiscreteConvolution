By Lemma~\ref{lemma_polynomial_p_and_odd_power} for every $m\in\mathbb{N}, \; n\in\mathbb{R}$
\begin{equation}
    \label{eq:current_a_def}
    n^{2m+1} = \sum_{r=0}^{m} \coeffA{m}{r} \sum_{k=0}^{n-1} k^r (n-k)^r
\end{equation}
The $\coeffA{m}{r}$ might be evaluated using binomial expansion of $\sum_{k=0}^{n-1} k^r (n-k)^r$
\begin{equation*}
    \sum_{k=0}^{n-1} k^r (n-k)^r
    =\sum_{k=0}^{n-1} k^r \sum_{j=0}^{r} (-1)^j \binom{r}{j} n^{r-j} k^{j}
    =\sum_{j=0}^{r} (-1)^j \binom{r}{j} n^{r-j} \sum_{k=0}^{n-1} k^{r+j}
\end{equation*}
Using Faulhaber's formula $\sum_{k=1}^{n} k^{p} = \frac{1}{p+1}\sum_{j=0}^{p} \binom{p+1}{j}
\bernoulli{j} n^{p+1-j}$ we get
\begin{equation}
    \label{eq:proof1}
    \begin{split}
        \sum_{k=0}^{n-1} k^r (n-k)^r
        &=\sum_{j=0}^{r} \binom{r}{j} n^{r-j} \frac{(-1)^j}{r+j+1}
        \left[\sum_{s} \binom{r+j+1}{s} \bernoulli{s} n^{r+j+1-s} - \bernoulli{r+j+1} \right] \\
        &=\sum_{j,s} \binom{r}{j} \frac{(-1)^j}{r+j+1} \binom{r+j+1}{s} \bernoulli{s} n^{2r+1-s}
        -\sum_{j} \binom{r}{j} \frac{(-1)^j}{r+j+1} \bernoulli{r+j+1} n^{r-j} \\
        &=\sum_{s} \underbrace{\sum_{j} \binom{r}{j} \frac{(-1)^j}{r+j+1} \binom{r+j+1}{s}}_{S(r)}
        \bernoulli{s} n^{2r+1-s} \\
        &-\sum_{j} \binom{r}{j} \frac{(-1)^j}{r+j+1} \bernoulli{r+j+1} n^{r-j}
    \end{split}
\end{equation}
where $\bernoulli{s}$ are Bernoulli numbers and $\bernoulli{1}=\frac{1}{2}$.
Now, we notice that
\begin{equation*}
    \sum_{j} \binom{r}{j} \frac{(-1)^j}{r+j+1} \binom{r+j+1}{s}
    =\begin{cases}
         \frac{1}{(2r+1) \binom{2r}r}, & \text{if } s=0;\\
         \frac{(-1)^r}{s} \binom{r}{2r-s+1}, & \text{if } s>0.
    \end{cases}
\end{equation*}
In particular, the last sum is zero for $0<s\leq r$.
Therefore, expression~\eqref{eq:proof1} takes the form
\begin{equation*}
    \begin{split}
        \sum_{k=0}^{n-1} k^r (n-k)^r
        &=\frac{1}{(2r+1) \binom{2r}{r}} n^{2r+1}
        +\underbrace{\sum_{s \geq 1} \frac{(-1)^r}{s} \binom{r}{2r-s+1} \bernoulli{s} n^{2r+1-s}}_{(\star)} \\
        &-\underbrace{\sum_{j} \binom{r}{j} \frac{(-1)^j}{r+j+1} \bernoulli{r+j+1} n^{r-j}}_{(\diamond)}
    \end{split}
\end{equation*}
Hence, introducing $\ell=2r+1-s$ to $(\star)$ and $\ell=r-j$ to $(\diamond)$, we get
\begin{equation*}
    \begin{split}
        \sum_{k=0}^{n-1} k^r (n-k)^r
        &=\frac{1}{(2r+1) \binom{2r}{r}} n^{2r+1}
        +\sum_{\ell} \frac{(-1)^r}{2r+1-\ell} \binom{r}{\ell} \bernoulli{2r+1-\ell} n^{\ell} \\
        &-\sum_{\ell} \binom{r}{\ell} \frac{(-1)^{j-\ell}}{2r+1-\ell} \bernoulli{2r+1-\ell} n^{\ell}
    \end{split}
\end{equation*}
\begin{equation*}
    \begin{split}
        \sum_{k=0}^{n-1} k^r (n-k)^r
        &=\frac{1}{(2r+1) \binom{2r}{r}} n^{2r+1}
        +(-1)^{r} \sum_{\ell} \frac{1}{2r+1-\ell} \binom{r}{\ell} \bernoulli{2r+1-\ell} n^{\ell} \\
        &-\frac{1}{(-1)^{r}} \sum_{\ell} \binom{r}{\ell} \frac{(-1)^{j-\ell}}{2r+1-\ell} \bernoulli{2r+1-\ell} n^{\ell} \\
        &=\frac{1}{(2r+1) \binom{2r}{r}}n^{2r+1}
        +2 \sum_{\text{odd } \ell}^{r} \frac{(-1)^r}{2r+1-\ell} \binom{r}{\ell} \bernoulli{2r+1-\ell} n^{\ell}
    \end{split}
\end{equation*}
Using the definition~\eqref{eq:current_a_def} of $\coeffA{m}{r}$, we obtain the following identity for polynomials in $n$
\begin{equation}
    \label{eq:proof2}
    \sum_{r=0}^{m} \coeffA{m}{r} \frac{1}{(2r+1) \binom{2r}{r}} n^{2r+1}
    +2 \sum_{r=0}^{m}\sum_{\text{odd } \ell}^{r} \coeffA{m}{r} \frac{(-1)^r}{2r+1-\ell}
    \binom{r}{\ell} \bernoulli{2r+1-\ell} n^{\ell}
    \equiv
    n^{2m+1}
\end{equation}
Taking the coefficient of $n^{2r+1}$ for $r=m$ in~\eqref{eq:proof2} we get $\coeffA{m}{m} = (2m+1) \binom{2m}{m}$.
Since that $\text{odd } \ell \leq r$ in explicit form is $2j + 1 \leq r$, it follows that $j \leq \frac{m-1}{2}$,
where $j$ is iterator.
Therefore, taking the coefficient of $n^{2j+1}$ for an integer $j$ in the range $\frac{m}{2} \leq j \leq m$,
we get $\coeffA{m}{j} = 0$.
Taking the coefficient of $n^{2d+1}$ for $d$ in the range $m/4 \leq d < m/2$ we get
\begin{equation*}
    \coeffA{m}{d} \frac{1}{(2d+1) \binom{2d}{d}}
    +2 (2m+1) \binom{2m}{m} \binom{m}{2d+1} \frac{(-1)^m}{2m-2d} \bernoulli{2m-2d} = 0,
\end{equation*}
i.e
\begin{equation*}
    \coeffA{m}{d} = (-1)^{m-1} \frac{(2m+1)!}{d!d!m!(m-2d-1)!} \frac{1}{m-d} \bernoulli{2m-2d}
\end{equation*}
Continue similarly we can express $\coeffA{m}{r}$ for each integer $r$ in range $m/2^{s+1}\leq r < m/2^s$
(iterating consecutively $s=1,2,\ldots$) via previously determined values of $\coeffA{m}{d}$ as follows
\begin{equation*}
    \coeffA{m}{r} =
    (2r+1) \binom{2r}{r} \sum_{d=2r+1}^{m} \coeffA{m}{d} \binom{d}{2r+1} \frac{(-1)^{d-1}}{d-r}
    \bernoulli{2d-2r}
\end{equation*}
