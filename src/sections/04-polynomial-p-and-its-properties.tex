\label{sec:polynomial-p-and-their-properties}
We continue our mathematical journey from the short overview
of polynomial $\polynomialL{m}{x}{k}$ which is
an essential part of polynomial $\polynomialP{m}{b}{x}$ since that
$\polynomialP{m}{b}{x} = \sum_{k=0}^{b-1} \polynomialL{m}{x}{k}$.
Polynomial $\polynomialL{m}{x}{k}$ is a polynomial of degree $2m$ in $x,k\in\mathbb{R}$,
see definition~\eqref{eq:def_polynomial_l}.
In its explicit form the polynomial $\polynomialL{m}{x}{k}$ is as follows
\begin{equation*}
    \polynomialL{m}{x}{k} =
    \coeffA{m}{m} k^m(x-k)^m +
    \coeffA{m}{m-1} k^{m-1}(x-k)^{m-1} +
    \cdots +
    \coeffA{m}{0}
\end{equation*}
where $\coeffA{m}{r}$ are real coefficients defined by~\eqref{eq:def_coeff_a}.
Coefficients $\coeffA{m}{r}$ are nonzero for $r$ only within the range $r \in \{m\} \cup \left[0,\frac{m-1}{2}\right]$.
For example,
\begin{table}[H]
    \setlength\extrarowheight{-6pt}
    \begin{tabular}{c|cccccccc}
        $m/r$ & 0 & 1       & 2      & 3      & 4   & 5    & 6     & 7     \\
        \hline
        0     & 1 &         &        &        &     &      &       &       \\
        1     & 1 & 6       &        &        &     &      &       &       \\
        2     & 1 & 0       & 30     &        &     &      &       &       \\
        3     & 1 & -14     & 0      & 140    &     &      &       &       \\
        4     & 1 & -120    & 0      & 0      & 630 &      &       &       \\
        5     & 1 & -1386   & 660    & 0      & 0   & 2772 &       &       \\
        6     & 1 & -21840  & 18018  & 0      & 0   & 0    & 12012 &       \\
        7     & 1 & -450054 & 491400 & -60060 & 0   & 0    & 0     & 51480
    \end{tabular}
    \caption{Coefficients $\coeffA{m}{r}$. See the OEIS entries
    \href{https://oeis.org/A302971}{\texttt{A302971}},
        \href{https://oeis.org/A304042}{\texttt{A304042}}: \cite{kolosov2018numerator, kolosov2018denominator}.}
    \label{tab:table_of_coefficients_a}
\end{table}
Thus, the polynomial $\polynomialL{m}{x}{k}$ may also be written as
\begin{equation*}
    \polynomialL{m}{x}{k} = \coeffA{m}{m} k^m (x-k)^m + \sum_{r=0}^{\frac{m-1}{2}} \coeffA{m}{r} k^r (x-k)^r
\end{equation*}
For example, the polynomials $\polynomialL{m}{x}{k}$ for $0\leq m\leq 3$ are
\begin{equation*}
    \begin{split}
        \polynomialL{0}{x}{k}
        &= 1, \\
        \polynomialL{1}{x}{k}
        &= 6 k (x-k) + 1
        = -6 k^2 + 6 k x + 1, \\
        \polynomialL{2}{x}{k}
        &=30 k^2 (x-k)^2+1
        =30 k^4-60 k^3 x+30 k^2 x^2+1, \\
        \polynomialL{3}{x}{k}
        &= 140 k^3 (x-k)^3-14 k (x-k)+1 \\
        &=-140 k^6+420 k^5 x-420 k^4 x^2+140 k^3 x^3+14 k^2-14 k x+1
    \end{split}
\end{equation*}
It is important to notice that $\polynomialL{m}{x}{k}$ is symmetric over $x$
\begin{ppty}
    \label{ppty_symmetry_of_polynomial_l}
    For every $x,k\in\mathbb{R}$
    \begin{equation*}
        \polynomialL{m}{x}{k} = \polynomialL{m}{x}{x-k}
    \end{equation*}
\end{ppty}
This might be seen from the following tables
\begin{table}[H]
    \setlength\extrarowheight{-6pt}
    \begin{tabular}{c|cccccccc}
        $x/k$ & 0 & 1  & 2  & 3  & 4  & 5  & 6  & 7 \\
        \hline
        0     & 1 &    &    &    &    &    &    &   \\
        1     & 1 & 1  &    &    &    &    &    &   \\
        2     & 1 & 7  & 1  &    &    &    &    &   \\
        3     & 1 & 13 & 13 & 1  &    &    &    &   \\
        4     & 1 & 19 & 25 & 19 & 1  &    &    &   \\
        5     & 1 & 25 & 37 & 37 & 25 & 1  &    &   \\
        6     & 1 & 31 & 49 & 55 & 49 & 31 & 1  &   \\
        7     & 1 & 37 & 61 & 73 & 73 & 61 & 37 & 1
    \end{tabular}
    ~\caption{Values of $\polynomialL{1}{x}{k}$.
    See the OEIS entry \href{https://oeis.org/A287326}{\texttt{A287326}}, \cite{kolosov2017third}.}
    \label{tab:fig_1}
\end{table}
Another case, given $m=2$ we have the following values of $\polynomialL{2}{x}{k}$
\begin{table}[H]
    \setlength\extrarowheight{-6pt}
    \begin{tabular}{c|cccccccc}
        $x/k$ & 0 & 1    & 2    & 3    & 4    & 5    & 6    & 7 \\
        \hline
        0     & 1 &      &      &      &      &      &      &   \\
        1     & 1 & 1    &      &      &      &      &      &   \\
        2     & 1 & 31   & 1    &      &      &      &      &   \\
        3     & 1 & 121  & 121  & 1    &      &      &      &   \\
        4     & 1 & 271  & 481  & 271  & 1    &      &      &   \\
        5     & 1 & 481  & 1081 & 1081 & 481  & 1    &      &   \\
        6     & 1 & 751  & 1921 & 2431 & 1921 & 751  & 1    &   \\
        7     & 1 & 1081 & 3001 & 4321 & 4321 & 3001 & 1081 & 1
    \end{tabular}
    \caption{Values of $\polynomialL{2}{x}{k}$.
    See the OEIS entry \href{https://oeis.org/A300656}{\texttt{A300656}}, ~\cite{kolosov2018fifth}.}
    \label{tab:row-sums-give-fifth-power}
\end{table}
Note that row sums of the table~\eqref{tab:fig_1} are cubes of $x$.
Next we discuss the polynomial $\polynomialP{m}{b}{x}$.
In its extended form, the polynomial $\polynomialP{m}{b}{x}$ is
\begin{equation*}
    \begin{split}
        \polynomialP{m}{b}{x} = \sum_{k=0}^{b-1} \polynomialL{m}{x}{k}
        =\sum_{k=0}^{b-1} \sum_{r=0}^{m} \coeffA{m}{r} k^r(x-k)^r
        =\sum_{r=0}^{m} \coeffA{m}{r} \sum_{k=0}^{b-1} k^r(x-k)^r
    \end{split}
\end{equation*}
By means of binomial theorem $(x-y)^n = \sum_{k=0}^{n} (-1)^{k} \binom{n}{k} x^{n-k} y^{k}$,
\begin{equation*}
    \begin{split}
        \polynomialP{m}{b}{x}
        &=\sum_{r=0}^{m} \coeffA{m}{r} \sum_{k=0}^{b-1} k^r \sum_{j=0}^{r} (-1)^{j} \binom{r}{j} x^{r-j} k^{j} \\
        &=\sum_{r=0}^{m} \coeffA{m}{r} \sum_{k=0}^{b-1} \sum_{j=0}^{r} (-1)^{j} \binom{r}{j} x^{r-j} k^{r+j} \\
        &=\sum_{r=0}^{m} \coeffA{m}{r} \sum_{j=0}^{r} (-1)^{j} x^{r-j} \binom{r}{j} \sum_{k=0}^{b-1} k^{r+j} \\
    \end{split}
\end{equation*}
However, by the symmetry~\eqref{ppty_symmetry_of_polynomial_l} of $\polynomialL{m}{x}{k}$ the polynomial
$\polynomialP{m}{b}{x}$ may also be written in the form
\begin{equation*}
    \begin{split}
        \polynomialP{m}{b}{x}
        &=\sum_{k=1}^{b} \sum_{r=0}^{m} \coeffA{m}{r} k^r(x-k)^r
        =\sum_{k=1}^{b} \sum_{r=0}^{m} \coeffA{m}{r} k^r \sum_{t=0}^{r} (-1)^{r-t} x^t \binom{r}{t} k^{r-t} \\
        &=\sum_{t=0}^{m} x^t
        \underbrace{\sum_{k=1}^{b} \sum_{r=t}^{m} (-1)^{r-t} \binom{r}{t} \coeffA{m}{r} k^{2r-t}}_{(-1)^{m-t} \polynomialX{m}{t}{b}}
    \end{split}
\end{equation*}
Note that
$\sum_{k=1}^{b} \sum_{r=t}^{m} (-1)^{r-t} \binom{r}{t} \coeffA{m}{r} k^{2r-t}$
is the
$(-1)^{m-t} \polynomialX{m}{t}{b}$.
From this formula it may be not immediately clear why $\polynomialX{m}{t}{b}$ represent polynomials in $b$.
However, this can be seen if we change the summation order and use Faulhaber's formula
$\sum_{k=1}^{n} k^{p}=\frac{1}{p+1}\sum _{j=0}^{p} \binom{p+1}{j} \bernoulli{j} n^{p+1-j}$
to obtain
\begin{equation*}
    \polynomialX{m}{t}{b} = (-1)^m \sum_{r=t}^{m} \binom{r}{t} \coeffA{m}{r} \frac{(-1)^r}{2r-t+1}
    \sum_{\ell=0}^{2r-t} \binom{2r-t+1}{\ell} \bernoulli{\ell} b^{2r-t+1-\ell}
\end{equation*}
Introducing $k=2r-t+1-\ell$ we further get the formula
\begin{equation*}
    \polynomialX{m}{t}{b} = (-1)^m \sum_{k=1}^{2m-t+1} b^k
    \underbrace{\sum_{r=t}^m \binom{r}{t} \coeffA{m}{r} \frac{(-1)^r}{2r-t+1} \binom{2r-t+1}{k}
    \bernoulli{2r-t+1-k}}_{\coeffH{m}{t}{k}}
\end{equation*}
Polynomials $\polynomialX{3}{t}{b}, \; 0\leq t \leq 3$ are
\begin{equation*}
    \begin{split}
        \polynomialX{3}{0}{j}
        &=7 b^2 - 28 b^3 + 70 b^5 - 70 b^6 + 20 b^7, \\
        \polynomialX{3}{1}{j}
        &=7 b - 42 b^2 + 175 b^4 - 210 b^5 + 70 b^6, \\
        \polynomialX{3}{2}{j}
        &=-14 b + 140 b^3 - 210 b^4 + 84 b^5, \\
        \polynomialX{3}{3}{j}
        &=35 b^2 - 70 b^3 + 35 b^4
    \end{split}
\end{equation*}
Polynomials $\coeffH{3}{t}{k}$ are defined by~\eqref{eq:def_coeff_h} and examples for $m=3, \; 0\leq t \leq 3$ are
\begin{equation*}
    \begin{split}
        \coeffH{3}{0}{k}
        &=\bernoulli{1-k} \binom{1}{k} + \frac{14}{3} \bernoulli{3-k} \binom{3}{k} - 20 \bernoulli{7 - k} \binom{7}{k}, \\
        \coeffH{3}{1}{k}
        &=7 \bernoulli{2-k} \binom{2}{k} - 70 \bernoulli{6-k} \binom{6}{k}, \\
        \coeffH{3}{2}{k}
        &=-84 \bernoulli{5-k} \binom{5}{k}, \\
        \coeffH{3}{3}{k}
        &=-35 \bernoulli{4-k} \binom{4}{k}
    \end{split}
\end{equation*}
It gives us an opportunity to overview the polynomial $\polynomialP{m}{b}{x}$ from the different prospective,
for instance
\begin{equation}
    \label{eq:p_all_forms}
    \polynomialP{m}{b}{x}
    =\sum_{r=0}^{m} (-1)^{m-r} \polynomialX{m}{r}{b} \cdot x^r
    =\sum_{r=0}^{m} \sum_{\ell=1}^{2m-r+1} (-1)^{2m-r} \coeffH{m}{r}{\ell} \cdot b^\ell \cdot x^r
\end{equation}
Equation~\eqref{eq:p_all_forms} clearly states why $\polynomialP{m}{b}{x}$ is polynomial in $x,b$.
For example,
\begin{equation*}
    \begin{split}
        \polynomialP{0}{b}{x}
        &=b, \\
        \polynomialP{1}{b}{x}
        &=3 b^2 - 2 b^3 - 3 b x + 3 b^2 x, \\
        \polynomialP{2}{b}{x}
        &=10 b^3 - 15 b^4 + 6 b^5 \\
        &- 15 b^2 x + 30 b^3 x - 15 b^4 x \\
        &+ 5 b x^2 - 15 b^2 x^2 + 10 b^3 x^2
    \end{split}
\end{equation*}
\begin{equation*}
    \begin{split}
        \polynomialP{3}{b}{x}
        &=-7 b^2 + 28 b^3 - 70 b^5 + 70 b^6 - 20 b^7 \\
        &+ 7 b x - 42 b^2 x + 175 b^4 x - 210 b^5 x + 70 b^6 x \\
        &+ 14 b x^2 - 140 b^3 x^2 + 210 b^4 x^2 - 84 b^5 x^2 \\
        &+ 35 b^2 x^3 - 70 b^3 x^3 + 35 b^4 x^3
    \end{split}
\end{equation*}
The following property is also true in terms of the polynomial $\polynomialP{m}{b}{x}$
\begin{ppty}
    \label{prop_p_identity}
    For every $m\in \mathbb{N}, \; x,b\in\mathbb{R}$
    \begin{equation*}
        \polynomialP{m}{b+1}{x} = \polynomialP{m}{b}{x} + \polynomialL{m}{x}{b}
    \end{equation*}
\end{ppty}
