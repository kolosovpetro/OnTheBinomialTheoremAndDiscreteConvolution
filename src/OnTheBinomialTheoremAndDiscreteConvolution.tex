\documentclass[12pt,letterpaper,oneside,reqno]{amsart}
\usepackage{amsfonts}
\usepackage{amsmath}
\usepackage{amssymb}
\usepackage{amsthm}
\usepackage{float}
\usepackage{mathrsfs}
\usepackage{colonequals}
\usepackage[font=small,labelfont=bf]{caption}
\usepackage[left=1in,right=1in,bottom=1in,top=1in]{geometry}
\usepackage[pdfpagelabels,hyperindex,colorlinks=true,linkcolor=blue,urlcolor=magenta,citecolor=green]{hyperref}
\usepackage{graphicx}
\linespread{1.7}
\emergencystretch=1em
\usepackage{array}
\usepackage{etoolbox}
\apptocmd{\sloppy}{\hbadness 10000\relax}{}{}
\raggedbottom

\newcommand \convPower [3]{(#1 \sp{#2} \ast #1 \sp{#2}) [#3]}
\newcommand \bernoulli [2][B] {{#1}\sb{#2}}
\newcommand \multifoldSum [2][x]{{#1}\sb{1} + {#1}\sb{2} + \cdots + {#1}\sb{#2}}
\newcommand \iversonBracket [1][s]{[#1 \; \mathrm{is} \; \mathrm{even}]}
\newcommand \floor [1]{\lfloor #1 \rfloor}
\newcommand \coeffA [3][A] {{\mathbf{#1}} \sb{#2,#3}}
\newcommand \coeffH [4][H] {{\mathbf{#1}} \sb{#2,#3} (#4)}
\newcommand \polynomialX [4][X] {{\mathbf{#1}}\sb{#2,#3} (#4)}
\newcommand \polynomialP [4][P]{{\mathbf{#1}}\sp{#2} \sb{#3}(#4)}
\newcommand \polynomialL [4][L]{{\mathbf{#1}}\sb{#2}(#3,#4)}
\newcommand \polynomialT [3][T]{{\mathbf{#1}}\sp{#2} (#3)}

\newtheorem{thm}{Theorem}[section]
\newtheorem{cor}[thm]{Corollary}
\newtheorem{prop}[thm]{Proposition}
\newtheorem{lem}[thm]{Lemma}
\newtheorem{ppty}[thm]{Property}
\newtheorem{quest}[thm]{Question}
\numberwithin{equation}{section}

%--------Meta Data: Fill in your info------
\title[On the link between binomial theorem and discrete convolution]
{On the link between binomial theorem and discrete convolution}
\author[Petro Kolosov]{Petro Kolosov}
\email{kolosovp94@gmail.com}
\keywords{Binomial theorem, Convolution, Discrete convolution, Polynomials}
\urladdr{https://kolosovpetro.github.io}
\subjclass[2010]{44A35 (primary), 11C08 (secondary)}
\date{\today}
\hypersetup{
    pdftitle={On the link between binomial theorem and discrete convolution},
    pdfsubject={Discrete Mathematics, Number Theory, Combinatorics},
    pdfauthor={Petro Kolosov},
    pdfkeywords={Binomial theorem, Discrete convolution, Power function, Polynomials, Convolution,
    Multinomial theorem, Binomial coefficient, Bernoulli number, Pascal's triangle, Faulhaber's formula,
    Power sum, Worpitzky identity, Binomial expansion}
}
\begin{document}
    \begin{abstract}
        Let $\mathbf{P}^{m}_{b}(x)$ be a $2m+1$-degree polynomial in $x,b\in\mathbb{R}$
\[
    \mathbf{P}^{m}_{b}(x) = \sum_{k=0}^{b-1} \sum_{r=0}^{m} \mathbf{A}_{m,r} k^r(x-k)^r
\]
where $\mathbf{A}_{m,r}$ is a real coefficient.
In this manuscript we establish a relation between Binomial theorem and polynomial $\mathbf{P}^{m}_{b}(x)$.
Furthermore, a relationship between Binomial theorem and discrete convolution in terms of polynomials is provided.

    \end{abstract}
    \maketitle
    \tableofcontents


    \section{Definitions, notations and conventions} \label{sec:definitions-notations-and-conventions}
    We now set the following notation, which remains fixed for the remainder of this manuscript
\begin{itemize}
    \setlength\itemsep{1.6em}
    \item $\coeffA{m}{r}$ is a real coefficient defined recursively
    \begin{equation}
        \label{eq:def_coeff_a}
        \coeffA{m}{r} =
        \begin{cases}
        (2r+1)
            \binom{2r}{r}, & \text{if } r=m; \\
            (2r+1) \binom{2r}{r} \sum_{d=2r+1}^{m} \coeffA{m}{d} \binom{d}{2r+1} \frac{(-1)^{d-1}}{d-r}
            \bernoulli{2d-2r}, & \text{if } 0 \leq r<m; \\
            0, & \text{if } r<0 \text{ or } r>m,
        \end{cases}
    \end{equation}
    where $m$ is non-negative integer and $\bernoulli{t}$ are Bernoulli numbers~\cite{WeissteinBernoulli}.
    It is assumed that $\bernoulli{1}=\frac{1}{2}$.

    \item $\polynomialP{m}{b}{x}$ is a $2m+1$-degree polynomial in $b,x\in\mathbb{R}$
    \begin{equation}
        \label{eq:def_polynomial_p}
        \polynomialP{m}{b}{x} = \sum_{k=0}^{b-1} \sum_{r=0}^{m} \coeffA{m}{r} k^r(x-k)^r
    \end{equation}

    \item $\coeffH{m}{t}{b}$ is a polynomial defined as
    \begin{equation}
        \label{eq:def_coeff_h}
        \coeffH{m}{t}{b}
        = \sum_{j=t}^{m} \binom{j}{t} \coeffA{m}{j} \frac{(-1)^j}{2j-t+1} \binom{2j-t+1}{b} \bernoulli{2j-t+1-b}
    \end{equation}
    integers $m,t,b$.

    \item $\polynomialX{m}{t}{j}$ is polynomial of degree $2m+1-t$ in $j\in\mathbb{R}$
    \begin{equation}
        \label{eq:def_coeff_x}
        \polynomialX{m}{t}{j} = (-1)^m \sum_{k=1}^{2m+1-t} \coeffH{m}{t}{k} \cdot j^k
    \end{equation}
    integers $m,t$.

    \item $\polynomialL{m}{x}{k}$ is $2m$ degree polynomial in $x,k\in\mathbb{R}$
    \begin{equation}
        \label{eq:def_polynomial_l}
        \polynomialL{m}{x}{k} = \sum_{r=0}^{m} \coeffA{m}{r} k^r(x-k)^r
    \end{equation}

    \item $(f\ast f)[n]$ is discrete convolution~\cite{damelin_discrete_convolution} of function $f$ defined over set of integers $\mathbb{Z}$
    \begin{align*}
    (f\ast f)[n]
        = \sum_{k} f(k) f(n-k)
    \end{align*}
\end{itemize}



    \section{Introduction and main results} \label{sec:introduction}
    The polynomial $\polynomialP{m}{b}{x}, \; m\in\mathbb{N}$ is $2m+1$-degree integer-valued polynomial in $x,b\in\mathbb{R}$.
\[
    \polynomialP{m}{b}{x} = \sum_{k=0}^{b-1} \sum_{r=0}^{m} \coeffA{m}{r} k^r(x-k)^r,
\]
where $\coeffA{m}{r}$ is real coefficient.
By means of Lemma~\ref{lemma_polynomial_p_and_odd_power},
the polynomial $\polynomialP{m}{b}{x}$ has the following relation with Binomial theorem~\cite{AbraSteg72}
\[
    \polynomialP{m}{x+y}{x+y} = \sum_{r=0}^{2m+1} \binom{2m+1}{r} x^{2m+1-r} y^r.
\]
From the other hand, polynomial $\polynomialP{m}{b}{x}$ might be expressed in terms of discrete convolution
of polynomial $n^j, \; j\in \mathbb{N}$
\[
    \polynomialP{m}{x+1}{x} = \sum_{r=0}^{m} \coeffA{m}{r} \convPower{n}{r}{x}, \quad n\geq 0.
\]
Therefore, it is easy to notice the following identities in terms of Binomial theorem and discrete convolution,
see Corollaries~\ref{cor_bin_exp_and_macaulay_conv},~\ref{cor_bin_exp_and_macaulay_conv_strict}
\begin{equation*}
    \sum_{r=0}^{m} \coeffA{m}{r} \convPower{n}{r}{x+y}
    =
    1 + \sum_{r=0}^{2m+1} \binom{2m+1}{r} x^{2m+1-r} y^r, \quad n \geq 0,
\end{equation*}
\begin{equation*}
    \sum_{r=0}^{m} \coeffA{m}{r} \convPower{n}{r}{x+y}
    =
    -1 + \sum_{r=0}^{2m+1} \binom{2m+1}{r} x^{2m+1-r} y^r, \quad n > 0.
\end{equation*}
Also, the following generalizations for multinomial case are discussed,
see Corollaries~\ref{cor_mult_exp_and_macaulay_conv},~\ref{cor_mult_exp_and_macaulay_conv_strict}
\begin{gather*}
    \sum_{r=0}^{m} \coeffA{m}{r} \convPower{n}{r}{\multifoldSum{t}} =
    1 + \sum_{\multifoldSum[k]{t}=2m+1} \binom{2m+1}{k_1, k_2,\ldots, k_t} \prod_{\ell=1}^{t} x_\ell^{k_\ell},
    \quad n \geq 0,\\
    \sum_{r=0}^{m} \coeffA{m}{r} \convPower{n}{r}{\multifoldSum{t}} =
    -1 + \sum_{\multifoldSum[k]{t}=2m+1} \binom{2m+1}{k_1, k_2,\ldots, k_t} \prod_{\ell=1}^{t} x_\ell^{k_\ell},
    \quad n > 0.\\
\end{gather*}
A few polynomial identities are straightforward as well,
Theorems~\ref{thm_odd_power_by_macaulays_convolution},~\ref{thm_odd_power_by_macaulays_convolution_strict}
\begin{gather*}
    x^{2m+1} = \sum_{r=0}^{m} \coeffA{m}{r} \sum_{k=0}^{x-1} k^r (x-k)^r,\\
    x^{2m+1} = \sum_{r=0}^{m} \coeffA{m}{r} \sum_{k=1}^{x} k^r (x-k)^r.\\
\end{gather*}



    \section{Polynomial \texorpdfstring{$\polynomialP{m}{b}{x}$}{P[m,b,x]} and its properties}
    \label{sec:polynomial-p-and-its-properties}
    \label{sec:polynomial-p-and-their-properties}
We continue our mathematical journey from short overview of polynomial $\polynomialL{m}{x}{k}$ that is
essential part of polynomial $\polynomialP{m}{b}{x}$ since that
$\polynomialP{m}{b}{x} = \sum_{k=0}^{b-1} \polynomialL{m}{x}{k}$.
Polynomial $\polynomialL{m}{x}{k}, \; m\in\mathbb{N}$ is polynomial of degree $2m$ in $x,k\in\mathbb{R}$,
see definition~\eqref{eq:def_polynomial_l}.
In explicit form the polynomial $\polynomialL{m}{x}{k}$ is as follows
\begin{equation*}
    \polynomialL{m}{x}{k} =
    \coeffA{m}{m} k^m(x-k)^m +
    \coeffA{m}{m-1} k^{m-1}(x-k)^{m-1} +
    \cdots +
    \coeffA{m}{0},
\end{equation*}
where $\coeffA{m}{r}$ are real coefficients defined by~\eqref{eq:def_coeff_a}.
Coefficients $\coeffA{m}{r}$ are nonzero only for $r$ within the interval $r \in \{m\} \cup \left[0,\frac{m-1}{2}\right]$.
For example,
\begin{table}[H]
    \begin{tabular}{c|cccccccc}
        $m/r$ & 0 & 1       & 2      & 3      & 4   & 5    & 6     & 7 \\ [3px]
        \hline
        0     & 1 &         &        &        &     &      &       &       \\
        1     & 1 & 6       &        &        &     &      &       &       \\
        2     & 1 & 0       & 30     &        &     &      &       &       \\
        3     & 1 & -14     & 0      & 140    &     &      &       &       \\
        4     & 1 & -120    & 0      & 0      & 630 &      &       &       \\
        5     & 1 & -1386   & 660    & 0      & 0   & 2772 &       &       \\
        6     & 1 & -21840  & 18018  & 0      & 0   & 0    & 12012 &       \\
        7     & 1 & -450054 & 491400 & -60060 & 0   & 0    & 0     & 51480
    \end{tabular}
    \caption{Coefficients $\coeffA{m}{r}$. See the OEIS entries:
    \href{https://oeis.org/A302971}{\texttt{A302971}},
        \href{https://oeis.org/A304042}{\texttt{A304042}}.} \label{tab:table_of_coefficients_a}
\end{table}
Thus, the polynomial $\polynomialL{m}{x}{k}$ may also be written as
\begin{equation*}
    \polynomialL{m}{x}{k} = \coeffA{m}{m} k^m (x-k)^m + \sum_{r=0}^{\frac{m-1}{2}} \coeffA{m}{r} k^r (x-k)^r
\end{equation*}
For example, the polynomials $\polynomialL{m}{x}{k}$ for $0\leq m\leq 3$ are
\begin{equation*}
    \begin{split}
        \polynomialL{0}{x}{k}
        &= 1, \\
        \polynomialL{1}{x}{k}
        &= 6 k (x-k) + 1
        = -6 k^2 + 6 k x + 1, \\
        \polynomialL{2}{x}{k}
        &=30 k^2 (x-k)^2+1
        =30 k^4-60 k^3 x+30 k^2 x^2+1, \\
        \polynomialL{3}{x}{k}
        &= 140 k^3 (x-k)^3-14 k (x-k)+1 \\
        &=-140 k^6+420 k^5 x-420 k^4 x^2+140 k^3 x^3+14 k^2-14 k x+1
    \end{split}
\end{equation*}
It is worth to notice that $\polynomialL{m}{x}{k}$ is symmetrical over $x$
\begin{ppty}
    \label{ppty_symmetry_of_polynomial_l}
    For every $x,k\in\mathbb{R}$
    \begin{equation*}
        \polynomialL{m}{x}{k} = \polynomialL{m}{x}{x-k}
    \end{equation*}
\end{ppty}
This might be seen in the following table
\begin{table}[H]
    \begin{tabular}{c|cccccccc}
        $x/k$ & 0 & 1  & 2  & 3  & 4  & 5  & 6  & 7 \\ [3px]
        \hline
        0     & 1 &    &    &    &    &    &    &   \\
        1     & 1 & 1  &    &    &    &    &    &   \\
        2     & 1 & 7  & 1  &    &    &    &    &   \\
        3     & 1 & 13 & 13 & 1  &    &    &    &   \\
        4     & 1 & 19 & 25 & 19 & 1  &    &    &   \\
        5     & 1 & 25 & 37 & 37 & 25 & 1  &    &   \\
        6     & 1 & 31 & 49 & 55 & 49 & 31 & 1  &   \\
        7     & 1 & 37 & 61 & 73 & 73 & 61 & 37 & 1 \\
    \end{tabular}
    \caption{Values of $\polynomialL{1}{x}{k}$. See the OEIS entry: \href{https://oeis.org/A287326}{\texttt{A287326}}.}
    \label{tab:fig_1}
\end{table}
Note that row sums of the table~\ref{tab:fig_1} are cubes of $x$.
Next we discuss the polynomial $\polynomialP{m}{b}{x}$.
In its extended form, the polynomial $\polynomialP{m}{b}{x}$ is
\begin{equation*}
    \begin{split}
        \polynomialP{m}{b}{x} = \sum_{k=0}^{b-1} \polynomialL{m}{x}{k}
        =\sum_{k=0}^{b-1} \sum_{r=0}^{m} \coeffA{m}{r} k^r(x-k)^r
        =\sum_{r=0}^{m} \coeffA{m}{r} \sum_{k=0}^{b-1} k^r(x-k)^r
    \end{split}
\end{equation*}
By the binomial theorem $(x-y)^n = \sum_{k=0}^{n} (-1)^{k} \binom{n}{k} x^{n-k} y^{k}$,
\begin{equation*}
    \begin{split}
        \polynomialP{m}{b}{x}
        &=\sum_{r=0}^{m} \coeffA{m}{r} \sum_{k=0}^{b-1} k^r \sum_{j=0}^{r} (-1)^{j} \binom{r}{j} x^{r-j} k^{j} \\
        &=\sum_{r=0}^{m} \coeffA{m}{r} \sum_{k=0}^{b-1} \sum_{j=0}^{r} (-1)^{j} \binom{r}{j} x^{r-j} k^{r+j} \\
        &=\sum_{r=0}^{m} \coeffA{m}{r} \sum_{j=0}^{r} (-1)^{j} x^{r-j} \binom{r}{j} \sum_{k=0}^{b-1} k^{r+j} \\
    \end{split}
\end{equation*}
However, by the symmetry~\eqref{ppty_symmetry_of_polynomial_l} of $\polynomialL{m}{x}{k}$ the polynomial
$\polynomialP{m}{b}{x}$ may also be written in the form
\begin{equation*}
    \begin{split}
        \polynomialP{m}{b}{x}
        &=\sum_{k=1}^{b} \sum_{r=0}^{m} \coeffA{m}{r} k^r(x-k)^r
        =\sum_{k=1}^{b} \sum_{r=0}^{m} \coeffA{m}{r} k^r \sum_{t=0}^{r} (-1)^{r-t} x^t \binom{r}{t} k^{r-t} \\
        &=\sum_{t=0}^{m} x^t
        \underbrace{\sum_{k=1}^{b} \sum_{r=t}^{m} (-1)^{r-t} \binom{r}{t} \coeffA{m}{r} k^{2r-t}}_{(-1)^{m-t} \polynomialX{m}{t}{b}}
    \end{split}
\end{equation*}
Note that
$\sum_{k=1}^{b} \sum_{r=t}^{m} (-1)^{r-t} \binom{r}{t} \coeffA{m}{r} k^{2r-t}$
is the
$(-1)^{m-t} \polynomialX{m}{t}{b}$.
From this formula it may be not immediately clear why $\polynomialX{m}{t}{b}$ represent polynomials in $b$.
However, this can be seen if we change the summation order and use Faulhaber's formula
$\sum_{k=1}^{n} k^{p}=\frac{1}{p+1}\sum _{j=0}^{p} \binom{p+1}{j} \bernoulli{j} n^{p+1-j}$
to obtain
\begin{equation*}
    \polynomialX{m}{t}{b} = (-1)^m \sum_{r=t}^{m} \binom{r}{t} \coeffA{m}{r} \frac{(-1)^r}{2r-t+1}
    \sum_{\ell=0}^{2r-t} \binom{2r-t+1}{\ell} \bernoulli{\ell} b^{2r-t+1-\ell}
\end{equation*}
Introducing $k=2r-t+1-\ell$ we further get the formula
\begin{equation*}
    \polynomialX{m}{t}{b} = (-1)^m \sum_{k=1}^{2m-t+1} b^k
    \underbrace{\sum_{r=t}^m \binom{r}{t} \coeffA{m}{r} \frac{(-1)^r}{2r-t+1} \binom{2r-t+1}{k}
    \bernoulli{2r-t+1-k}}_{\coeffH{m}{t}{k}}
\end{equation*}
Polynomials $\polynomialX{3}{t}{b}, \; 0\leq t \leq 3$ are
\begin{equation*}
    \begin{split}
        \polynomialX{3}{0}{j}
        &=7 b^2 - 28 b^3 + 70 b^5 - 70 b^6 + 20 b^7, \\
        \polynomialX{3}{1}{j}
        &=7 b - 42 b^2 + 175 b^4 - 210 b^5 + 70 b^6, \\
        \polynomialX{3}{2}{j}
        &=-14 b + 140 b^3 - 210 b^4 + 84 b^5, \\
        \polynomialX{3}{3}{j}
        &=35 b^2 - 70 b^3 + 35 b^4
    \end{split}
\end{equation*}
Polynomials $\coeffH{3}{t}{k}$ are defined by~\eqref{eq:def_coeff_h} and examples for $m=3, \; 0\leq t \leq 3$ are
\begin{equation*}
    \begin{split}
        \coeffH{3}{0}{k}
        &=\bernoulli{1-k} \binom{1}{k} + \frac{14}{3} \bernoulli{3-k} \binom{3}{k} - 20 \bernoulli{7 - k} \binom{7}{k}, \\
        \coeffH{3}{1}{k}
        &=7 \bernoulli{2-k} \binom{2}{k} - 70 \bernoulli{6-k} \binom{6}{k}, \\
        \coeffH{3}{2}{k}
        &=-84 \bernoulli{5-k} \binom{5}{k}, \\
        \coeffH{3}{3}{k}
        &=-35 \bernoulli{4-k} \binom{4}{k}
    \end{split}
\end{equation*}
It gives us an opportunity to overview the polynomial $\polynomialP{m}{b}{x}$ from the different prospective,
for instance
\begin{equation}
    \label{eq:p_all_forms}
    \polynomialP{m}{b}{x}
    =\sum_{r=0}^{m} (-1)^{m-r} \polynomialX{m}{r}{b} \cdot x^r
    =\sum_{r=0}^{m} \sum_{\ell=1}^{2m-r+1} (-1)^{2m-r} \coeffH{m}{r}{\ell} \cdot b^\ell \cdot x^r
\end{equation}
Equation~\eqref{eq:p_all_forms} clearly states why $\polynomialP{m}{b}{x}$ is polynomial in $x,b$.
For example,
\begin{equation*}
    \begin{split}
        \polynomialP{0}{b}{x}
        &=b, \\
        \polynomialP{1}{b}{x}
        &=3 b^2 - 2 b^3 - 3 b x + 3 b^2 x, \\
        \polynomialP{2}{b}{x}
        &=10 b^3 - 15 b^4 + 6 b^5 \\
        &- 15 b^2 x + 30 b^3 x - 15 b^4 x \\
        &+ 5 b x^2 - 15 b^2 x^2 + 10 b^3 x^2
    \end{split}
\end{equation*}
\begin{equation*}
    \begin{split}
        \polynomialP{3}{b}{x}
        &=-7 b^2 + 28 b^3 - 70 b^5 + 70 b^6 - 20 b^7 \\
        &+ 7 b x - 42 b^2 x + 175 b^4 x - 210 b^5 x + 70 b^6 x \\
        &+ 14 b x^2 - 140 b^3 x^2 + 210 b^4 x^2 - 84 b^5 x^2 \\
        &+ 35 b^2 x^3 - 70 b^3 x^3 + 35 b^4 x^3
    \end{split}
\end{equation*}
The following property is also true in terms of $\polynomialP{m}{b}{x}$
\begin{ppty}
    \label{prop_p_identity}
    For every $m\in \mathbb{N}, \; x,b\in\mathbb{R}$
    \begin{equation*}
        \polynomialP{m}{b+1}{x} = \polynomialP{m}{b}{x} + \polynomialL{m}{x}{b}
    \end{equation*}
\end{ppty}



    \section{Polynomial \texorpdfstring{$\polynomialP{m}{b}{x}$}{P[m,b,x]} in terms of Binomial theorem}
    \label{sec:odd-binomial-expansion-as-partial-case-of-polynomial-p}
    \begin{lem}
    \label{lemma_polynomial_p_and_odd_power}
    For every $m\in\mathbb{N}, \; x,y\in\mathbb{R}$
    \begin{equation*}
        \polynomialP{m}{x+y}{x+y} = \sum_{r=0}^{2m+1} \binom{2m+1}{r} x^{2m+1-r} y^r
    \end{equation*}
\end{lem}
By Lemma~\ref{lemma_polynomial_p_and_odd_power} and equation~\eqref{eq:p_all_forms} the following
polynomial identities straightforward
\begin{equation*}
    x^{2m+1}
    =\sum_{r=0}^{m} \sum_{\ell=1}^{2m-r+1} (-1)^{2m-r} \coeffH{m}{r}{\ell} \cdot x^{\ell+r}
    =\sum_{r=0}^{m} (-1)^{m-r} \polynomialX{m}{r}{x} \cdot x^r
\end{equation*}
For instance,
\begin{equation*}
    \polynomialP{2}{x+y}{x+y} = (x + y) (x^4 + 4 x^3 y + 6 x^2 y^2 + 4 x y^3 + y^4).
\end{equation*}
In addition, the following identities hold
\begin{equation*}
    \begin{split}
    (x+y)
        ^{2m+1}
        &=\sum_{r=0}^{m} \sum_{\ell=1}^{2m-r+1} (-1)^{2m-r} \coeffH{m}{r}{\ell} \cdot (x+y)^{\ell+r} \\
        &=\sum_{r=0}^{m} (-1)^{m-r} \polynomialX{m}{r}{x+y} \cdot (x+y)^r
    \end{split}
\end{equation*}
Obviously, Multinomial expansion of $t$-fold sum $(\multifoldSum{t})^{2m+1}$ can be reached
by $\polynomialP{m}{b}{\multifoldSum{t}}$ as well
\begin{cor}
    For all $x_1,x_2,\ldots, x_t\in\mathbb{R}, \; m \in \mathbb{N}$
    \begin{equation*}
        \polynomialP{m}{\multifoldSum{t}}{\multifoldSum{t}}
        =
        \sum_{\multifoldSum[k]{t}=2m+1} \binom{2m+1}{k_1, k_2,\ldots, k_t} \prod_{s=1}^{t} x_t^{k_s}
    \end{equation*}
\end{cor}
Moreover, the following multinomial identities hold
\begin{equation*}
    \begin{split}
    (\multifoldSum{t})
        ^{2m+1}
        &=\sum_{r=0}^{m} \sum_{\ell=1}^{2m-r+1} (-1)^{2m-r} \coeffH{m}{r}{\ell} \cdot (\multifoldSum{t})^{\ell+r} \\
        &=\sum_{r=0}^{m} (-1)^{m-r} \polynomialX{m}{r}{\multifoldSum{t}} \cdot (\multifoldSum{t})^r
    \end{split}
\end{equation*}


    \section{Polynomial \texorpdfstring{$\polynomialP{m}{b}{x}$}{P[m,b,x]} in terms of Discrete convolution}
    \label{sec:relation-between-p-and-convolution-of-polynomials}
    In this section we discuss the relation between $\polynomialP{m}{b}{x}$ and discrete convolution of
polynomials.
To show that $\polynomialP{m}{b}{x}$ involves the discrete convolution of polynomial $n^r$
let's remind the definition of $\polynomialP{m}{b}{x}$
\begin{equation*}
    \polynomialP{m}{b}{x} = \sum_{k=0}^{b-1} \sum_{r=0}^{m} \coeffA{m}{r} k^r (x-k)^r
    = \sum_{r=0}^{m} \coeffA{m}{r} \sum_{k=0}^{b-1} k^r (x-k)^r
\end{equation*}
A discrete convolution of defined over set of integers $\mathbb{Z}$ function $f$ is
\begin{equation*}
(f \ast f)[n]
    = \sum_{k} f(k) f(n-k)
\end{equation*}
General formula of discrete convolution for polynomials $f(n) = n^j, \; n\geq a \in \mathbb{R}$ may be derived immediately
\begin{equation*}
    \begin{split}
        \convPower{n}{j}{x}
        &=\sum_{k} k^j (x-k)^j [k\geq a][x-k\geq a] \\
        &=\sum_{k} k^j (x-k)^j [k\geq a][k\leq x-a] \\
        &=\sum_{k} k^j (x-k)^j [a \leq k \leq x-a] \\
        &=\sum_{k=a}^{x-a} k^j (x-k)^j,
    \end{split}
\end{equation*}
where $[a \leq k \leq x-a]$ is Iverson's bracket~\cite{APL}.
\begin{lem}
    \label{lemma_disc_conv_identity}
    For every $n\in\mathbb{N}, \; x\in\mathbb{R}$
    \[
        \convPower{n}{r}{x} = \sum_{k=0}^{x} k^r (x-k)^r, \quad n\geq 0.
    \]
\end{lem}
Thus, the corollary follows
\begin{cor}
    \label{cor_polynomial_p_and_macaulay_convolution}
    By Lemma~\ref{lemma_disc_conv_identity} the polynomial $\polynomialP{m}{b}{n}$ might be expressed in terms
    of discrete convolution as follows
    \[
        \polynomialP{m}{x+1}{x} = \sum_{r=0}^{m} \coeffA{m}{r} \convPower{n}{r}{x}, \quad n\geq 0.
    \]
\end{cor}
Therefore, another polynomial identity follows
\begin{thm}
    \label{thm_odd_power_by_macaulays_convolution}
    By Lemma~\ref{lemma_polynomial_p_and_odd_power}, Corollary~\ref{cor_polynomial_p_and_macaulay_convolution}
    and property~\ref{prop_p_identity}, for every $m\in\mathbb{N}, \; x\in\mathbb{R}$
    \begin{equation*}
        x^{2m+1} = -1 + \sum_{r=0}^{m} \coeffA{m}{r} \convPower{n}{r}{x}, \quad n\geq 0.
    \end{equation*}
\end{thm}
Now we notice the following identity in terms of polynomial $\polynomialP{m}{b}{x}$ and
discrete convolution $\convPower{n}{j}{x}$
\begin{prop}
    \label{prop_polynomial_p_and_macaulay_convolution_strict}
    For every $m \in \mathbb{N}, \; x\in\mathbb{R}$
    \begin{equation*}
        \begin{split}
            \polynomialP{m}{x}{x}
            &=\sum_{r=0}^{m} \coeffA{m}{r} \left(0^r x^r + \sum_{k=1}^{x-1} k^r (x-k)^r \right) \\
            &=\sum_{r=0}^{m} \coeffA{m}{r} 0^r x^r + \sum_{r=0}^{m} \coeffA{m}{r} \convPower{n}{r}{x} \\
            &=1 + \sum_{r=0}^{m} \coeffA{m}{r} \convPower{n}{r}{x}, \quad n\geq 1.
        \end{split}
    \end{equation*}
\end{prop}
Since that for all $r$ in $\coeffA{m}{r} 0^r x^r$ we have
\begin{equation*}
    \coeffA{m}{r} 0^r x^r =
    \begin{cases}
        1, & \mbox{if } r=0 \\
        0, & \mbox{if } r>0
    \end{cases}
\end{equation*}
Above is true because $\coeffA{m}{0}=1$ for every $m\in\mathbb{N}$, and $x^0 = 1$
for every $x$, ~\cite{Grah94SN}.
Hence, the following identity between $\polynomialP{m}{b}{x}$ and
discrete convolution $\convPower{n}{j}{x}$ holds
\begin{thm}
    \label{thm_odd_power_by_macaulays_convolution_strict}
    By Lemma~\ref{lemma_polynomial_p_and_odd_power} and
    Proposition~\ref{prop_polynomial_p_and_macaulay_convolution_strict},
    for every $m\in\mathbb{N}, \; x\in\mathbb{R}$
    \begin{equation*}
        x^{2m+1} = 1 + \sum_{r=0}^{m} \coeffA{m}{r} \convPower{n}{r}{x}, \quad n\geq 1.
    \end{equation*}
\end{thm}
\begin{cor}
    \label{cor_sum_of_coeffs_a}
    By Theorem~\ref{thm_odd_power_by_macaulays_convolution_strict}, for all $m\in\mathbb{N}$
    \begin{equation*}
        \sum_{r=0}^{m} \coeffA{m}{r} = 2^{2m+1} - 1
    \end{equation*}
\end{cor}
Corollary~\ref{cor_sum_of_coeffs_a} holds since that convolution $\convPower{n}{j}{x}=1, \; n\geq 1$
for each $r$ and $x=2$.



    \section{Relation between Binomial theorem and Discrete convolution}
    \label{sec:relation-between-binomial-theorem-and-discrete-convolution}
    \begin{cor}
    \label{cor_bin_exp_and_macaulay_conv}
    (Generalization of Theorem~\ref{thm_odd_power_by_macaulays_convolution} for Binomials.)
    For every $m\in\mathbb{N}, \; x,y\in\mathbb{R}$
    \begin{equation*}
        \sum_{r=0}^{m} \coeffA{m}{r} \convPower{n}{r}{x+y}
        =
        1 + \sum_{r=0}^{2m+1} \binom{2m+1}{r} x^{2m+1-r} y^r, \quad n\geq 1.
    \end{equation*}
\end{cor}
For example, given $m=0,1,2$ the Corollary~\ref{cor_bin_exp_and_macaulay_conv} gives
\begin{equation*}
    \begin{split}
        \sum_{r=0}^{0} \coeffA{0}{r} \convPower{n}{r}{x+y}
        &= 1 + x + y \\
        \sum_{r=0}^{1} \coeffA{1}{r} \convPower{n}{r}{x+y}
        &= 1 + x + y - (x + y) (1 + x + y) (1 - 3 x - 3 y + 2 (x + y)) \\
        &= x^3 + 3 x^2 y + 3 x y^2 + y^3 + 1\\
        \sum_{r=0}^{2} \coeffA{2}{r} \convPower{n}{r}{x+y}
        &=1 + x + y + (x + y) (1 + x + y) \left(-1 + x + 5 x^2 + y + 10 x y + 5 y^2\right. \\
        &-15 x (x + y) + 10 x^2 (x + y) - 15 y (x + y) + 20 x y (x + y) \\
        &+ 10 y^2 (x + y) +9 (x + y)^2 - 15 x (x + y)^2 \\
        &\left.-15 y (x + y)^{2} + 6 {(x + y)}^{3}\right) \\
        &=x^5 + 5 x^4 y + 10 x^3 y^2 + 10 x^2 y^3 + 5 x y^4 + y^5 + 1
    \end{split}
\end{equation*}
To get above examples use \texttt{ConvPowerIdentity[m, x + y]} command in Mathematica console~\cite{mmca_package}.
\begin{cor}
    \label{cor_bin_exp_and_macaulay_conv_strict}
    (Generalization of Theorem~\ref{thm_odd_power_by_macaulays_convolution_strict} for Binomials.)
    For every $m\in\mathbb{N}, \; x,y\in\mathbb{R}$
    \begin{equation*}
        \sum_{r=0}^{m} \coeffA{m}{r} \convPower{n}{r}{x+y}
        =
        -1 + \sum_{r=0}^{2m+1} \binom{2m+1}{r} x^{2m+1-r} y^r, \quad n\geq 0.
    \end{equation*}
\end{cor}
For example, given $m=0,1$ the Corollary~\ref{cor_bin_exp_and_macaulay_conv_strict} gives
\begin{equation*}
    \begin{split}
        \sum_{r=0}^{0} \coeffA{0}{r} \convPower{n}{r}{x+y}
        &= x + y - 1 \\
        \sum_{r=0}^{1} \coeffA{1}{r} \convPower{n}{r}{x+y}
        &= -1 + x + y - (-1 + x + y) (x + y) (-1 - 3 x - 3 y + 2 (x + y)) \\
        &= x^3 + 3 x^2 y + 3 x y^2 + y^3 - 1
    \end{split}
\end{equation*}
To get above examples use \texttt{ConvPowerIdentityStrict[m, x + y]} command in Mathematica console~\cite{mmca_package}.
From other prospective, let be a function $f_r(t,k) = (t-k)^r, \; t \geq k$, then following identity holds
\begin{equation}
(x-2a)
    ^{2m+1} + 1 =\sum_{r=0}^{m} \coeffA{m}{r} (f_r(t,k) \ast f_r(t,k))[x]
    \label{eq:parametric-identity}
\end{equation}
Let be a function $g_r(t,k) = (t-k)^r, \; t > k$, then
\begin{equation}
(x-2a)
    ^{2m+1} - 1 =\sum_{r=0}^{m} \coeffA{m}{r} (g_r(t,k) \ast g_r(t,k))[x]
    \label{eq:parametric-identity-strict}
\end{equation}

\subsection{Generalization for Multinomials} \label{subsec:generalization-for-multinomials}
In this subsection we generalize
Theorems~\ref{thm_odd_power_by_macaulays_convolution},~\ref{thm_odd_power_by_macaulays_convolution_strict}
for multinomial cases.
\begin{cor}
    \label{cor_mult_exp_and_macaulay_conv}
    (Generalization of Theorem~\ref{thm_odd_power_by_macaulays_convolution} for Multinomials.)
    For every $x_1, x_2, \ldots, x_t\in\mathbb{R}, \; m\in\mathbb{N}, \; n\geq1\in\mathbb{N}$
    \[
        \sum_{r=0}^{m} \coeffA{m}{r} \convPower{n}{r}{\multifoldSum{t}} =
        1 + \sum_{\multifoldSum[k]{t}=2m+1} \binom{2m+1}{k_1, k_2,\ldots, k_t} \prod_{\ell=1}^{t} x_\ell^{k_\ell}
    \]
\end{cor}
For instance, given $m=1$ the Corollary~\ref{cor_mult_exp_and_macaulay_conv} gives
\begin{equation*}
    \begin{split}
        &\sum_{r=0}^{1} \coeffA{1}{r} \convPower{n}{r}{x+y+z} \\
        &=1 + x + y + z - (x + y + z) (1 + x + y + z) (1 - 3 x - 3 y - 3 z + 2 (x + y + z)) \\
        &=1 + x^3 + 3 x^2 y + 3 x y^2 + y^3 + 3 x^2 z + 6 x y z + 3 y^2 z + 3 x z^2 + 3 y z^2 + z^3,
    \end{split}
\end{equation*}
it might be verified using  \texttt{ConvPowerIdentity[m, x + y + z]}
command in Mathematica console~\cite{mmca_package}.
\begin{cor}
    \label{cor_mult_exp_and_macaulay_conv_strict}
    (Generalization of Theorem~\ref{thm_odd_power_by_macaulays_convolution_strict} for Multinomials.)
    For each $\multifoldSum{t} \geq 1, \; x_1,x_2,\ldots,x_t\in\mathbb{R}, \; m\in\mathbb{N}, \; n\geq1\in\mathbb{N}$
    \[
        \sum_{r=0}^{m} \coeffA{m}{r} \convPower{n}{r}{\multifoldSum{t}} =
        -1 + \sum_{\multifoldSum[k]{t}=2m+1} \binom{2m+1}{k_1, k_2,\ldots, k_t} \prod_{\ell=1}^{t} x_\ell^{k_\ell}
    \]
\end{cor}
For example, given $m=1$ the Corollary~\ref{cor_mult_exp_and_macaulay_conv_strict} gives
\begin{equation*}
    \begin{split}
        &\sum_{r=0}^{1} \coeffA{1}{r} \convPower{n}{r}{x+y+z} \\
        &=x + y + z -1 - (x + y + z -1) (x + y + z) (2 (x + y + z) - 1 - 3 x - 3 y - 3 z) \\
        &=x^3 + 3 x^2 y + 3 x y^2 + y^3 + 3 x^2 z + 6 x y z + 3 y^2 z + 3 x z^2 + 3 y z^2 + z^3-1,
    \end{split}
\end{equation*}
it might be verified using \texttt{ConvPowerIdentityStrict[m, x + y + z]}
command in Mathematica console~\cite{mmca_package}.


    \section{Derivation of coefficient \texorpdfstring{$\coeffA{m}{r}$}{A[m,r]}}
    \label{sec:derivation-of-coefficients-a}
    By Lemma~\ref{lemma_polynomial_p_and_odd_power} for every $m\in\mathbb{N}, \; n\in\mathbb{R}$
\begin{equation}
    \label{eq:current_a_def}
    n^{2m+1} = \sum_{r=0}^{m} \coeffA{m}{r} \sum_{k=0}^{n-1} k^r (n-k)^r
\end{equation}
The $\coeffA{m}{r}$ might be evaluated using binomial expansion of $\sum_{k=0}^{n-1} k^r (n-k)^r$
\begin{equation*}
    \sum_{k=0}^{n-1} k^r (n-k)^r
    =\sum_{k=0}^{n-1} k^r \sum_{j=0}^{r} (-1)^j \binom{r}{j} n^{r-j} k^{j}
    =\sum_{j=0}^{r} (-1)^j \binom{r}{j} n^{r-j} \sum_{k=0}^{n-1} k^{r+j}
\end{equation*}
Using Faulhaber's formula $\sum_{k=1}^{n} k^{p} = \frac{1}{p+1}\sum_{j=0}^{p} \binom{p+1}{j}
\bernoulli{j} n^{p+1-j}$ we get
\begin{equation}
    \label{eq:proof1}
    \begin{split}
        \sum_{k=0}^{n-1} k^r (n-k)^r
        &=\sum_{j=0}^{r} \binom{r}{j} n^{r-j} \frac{(-1)^j}{r+j+1}
        \left[\sum_{s} \binom{r+j+1}{s} \bernoulli{s} n^{r+j+1-s} - \bernoulli{r+j+1} \right] \\
        &=\sum_{j,s} \binom{r}{j} \frac{(-1)^j}{r+j+1} \binom{r+j+1}{s} \bernoulli{s} n^{2r+1-s}
        -\sum_{j} \binom{r}{j} \frac{(-1)^j}{r+j+1} \bernoulli{r+j+1} n^{r-j} \\
        &=\sum_{s} \underbrace{\sum_{j} \binom{r}{j} \frac{(-1)^j}{r+j+1} \binom{r+j+1}{s}}_{S(r)}
        \bernoulli{s} n^{2r+1-s} \\
        &-\sum_{j} \binom{r}{j} \frac{(-1)^j}{r+j+1} \bernoulli{r+j+1} n^{r-j}
    \end{split}
\end{equation}
where $\bernoulli{s}$ are Bernoulli numbers and $\bernoulli{1}=\frac{1}{2}$.
Now, we notice that
\begin{equation*}
    \sum_{j} \binom{r}{j} \frac{(-1)^j}{r+j+1} \binom{r+j+1}{s}
    =\begin{cases}
         \frac{1}{(2r+1) \binom{2r}r}, & \text{if } s=0;\\
         \frac{(-1)^r}{s} \binom{r}{2r-s+1}, & \text{if } s>0.
    \end{cases}
\end{equation*}
In particular, the last sum is zero for $0<s\leq r$.
Therefore, expression~\eqref{eq:proof1} takes the form
\begin{equation*}
    \begin{split}
        \sum_{k=0}^{n-1} k^r (n-k)^r
        &=\frac{1}{(2r+1) \binom{2r}{r}} n^{2r+1}
        +\underbrace{\sum_{s \geq 1} \frac{(-1)^r}{s} \binom{r}{2r-s+1} \bernoulli{s} n^{2r+1-s}}_{(\star)} \\
        &-\underbrace{\sum_{j} \binom{r}{j} \frac{(-1)^j}{r+j+1} \bernoulli{r+j+1} n^{r-j}}_{(\diamond)}
    \end{split}
\end{equation*}
Hence, introducing $\ell=2r+1-s$ to $(\star)$ and $\ell=r-j$ to $(\diamond)$, we get
\begin{equation*}
    \begin{split}
        \sum_{k=0}^{n-1} k^r (n-k)^r
        &=\frac{1}{(2r+1) \binom{2r}{r}} n^{2r+1}
        +\sum_{\ell} \frac{(-1)^r}{2r+1-\ell} \binom{r}{\ell} \bernoulli{2r+1-\ell} n^{\ell} \\
        &-\sum_{\ell} \binom{r}{\ell} \frac{(-1)^{j-\ell}}{2r+1-\ell} \bernoulli{2r+1-\ell} n^{\ell}
    \end{split}
\end{equation*}
\begin{equation*}
    \begin{split}
        \sum_{k=0}^{n-1} k^r (n-k)^r
        &=\frac{1}{(2r+1) \binom{2r}{r}} n^{2r+1}
        +(-1)^{r} \sum_{\ell} \frac{1}{2r+1-\ell} \binom{r}{\ell} \bernoulli{2r+1-\ell} n^{\ell} \\
        &-\frac{1}{(-1)^{r}} \sum_{\ell} \binom{r}{\ell} \frac{(-1)^{j-\ell}}{2r+1-\ell} \bernoulli{2r+1-\ell} n^{\ell} \\
        &=\frac{1}{(2r+1) \binom{2r}{r}}n^{2r+1}
        +2 \sum_{\text{odd } \ell}^{r} \frac{(-1)^r}{2r+1-\ell} \binom{r}{\ell} \bernoulli{2r+1-\ell} n^{\ell}
    \end{split}
\end{equation*}
Using the definition~\eqref{eq:current_a_def} of $\coeffA{m}{r}$, we obtain the following identity for polynomials in $n$
\begin{equation}
    \label{eq:proof2}
    \sum_{r=0}^{m} \coeffA{m}{r} \frac{1}{(2r+1) \binom{2r}{r}} n^{2r+1}
    +2 \sum_{r=0}^{m}\sum_{\text{odd } \ell}^{r} \coeffA{m}{r} \frac{(-1)^r}{2r+1-\ell}
    \binom{r}{\ell} \bernoulli{2r+1-\ell} n^{\ell}
    \equiv
    n^{2m+1}
\end{equation}
Taking the coefficient of $n^{2r+1}$ for $r=m$ in~\eqref{eq:proof2} we get $\coeffA{m}{m} = (2m+1) \binom{2m}{m}$.
Since that $\text{odd } \ell \leq r$ in explicit form is $2j + 1 \leq r$, it follows that $j \leq \frac{m-1}{2}$,
where $j$ is iterator.
Therefore, taking the coefficient of $n^{2j+1}$ for an integer $j$ in the range $\frac{m}{2} \leq j \leq m$,
we get $\coeffA{m}{j} = 0$.
Taking the coefficient of $n^{2d+1}$ for $d$ in the range $m/4 \leq d < m/2$ we get
\begin{equation*}
    \coeffA{m}{d} \frac{1}{(2d+1) \binom{2d}{d}}
    +2 (2m+1) \binom{2m}{m} \binom{m}{2d+1} \frac{(-1)^m}{2m-2d} \bernoulli{2m-2d} = 0,
\end{equation*}
i.e
\begin{equation*}
    \coeffA{m}{d} = (-1)^{m-1} \frac{(2m+1)!}{d!d!m!(m-2d-1)!} \frac{1}{m-d} \bernoulli{2m-2d}
\end{equation*}
Continue similarly we can express $\coeffA{m}{r}$ for each integer $r$ in range $m/2^{s+1}\leq r < m/2^s$
(iterating consecutively $s=1,2,\ldots$) via previously determined values of $\coeffA{m}{d}$ as follows
\begin{equation*}
    \coeffA{m}{r} =
    (2r+1) \binom{2r}{r} \sum_{d=2r+1}^{m} \coeffA{m}{d} \binom{d}{2r+1} \frac{(-1)^{d-1}}{d-r}
    \bernoulli{2d-2r}
\end{equation*}
So that
\begin{align*}
    \coeffA{m}{r} =
    \begin{cases}
    (2r+1)
        \binom{2r}{r}, & \text{if } r=m; \\
        (2r+1) \binom{2r}{r} \sum_{d=2r+1}^{m} \coeffA{m}{d} \binom{d}{2r+1} \frac{(-1)^{d-1}}{d-r}
        \bernoulli{2d-2r}, & \text{if } 0 \leq r<m; \\
        0, & \text{if } r<0 \text{ or } r>m,
    \end{cases}
\end{align*}



    \section{Verification of the results and examples}
    \label{sec:verification-of-the-results-and-examples}
    To fulfill our study we provide an opportunity to verify its results by means of Wolfram Mathematica language.

\subsection{Mathematica commands} \label{subsec:mathematica-commands}
Proceeding to the repository~\cite{PK22Source} the following commands to verify the formulas:
\begin{itemize}
    \item \texttt{A[m, r]} returns the real coefficient $\coeffA{m}{r}$ defined by~\eqref{eq:def_coeff_a}.
    \item \texttt{PolynomialL[m, n, k]} returns the polynomial $\polynomialL{m}{n}{k}$ defined by~\eqref{eq:def_polynomial_l}.
    \item \texttt{PolynomialP[m, x, b]} returns the polynomial $\polynomialP{m}{x}{b}$ defined by~\eqref{eq:def_polynomial_p}.
    \item \texttt{Expand[PolynomialP[m, x + y, x + y]]} verifies the Lemma~\ref{lemma_polynomial_p_and_odd_power}.
    \item \texttt{PolynomialH[m, t, j]} returns the polynomial $\coeffH{m}{t}{j}$ defined by~\eqref{eq:def_coeff_h}.
    \item \texttt{PolynomialX[m, t, k]} returns the polynomial $\polynomialX{m}{t}{k}$ defined by~\eqref{eq:def_coeff_x}.
    \item \texttt{Expand[BinomialTheoremAndDiscreteConvolutionTest[m, x + y]]} verifies the Corollary~\ref{cor_bin_exp_and_macaulay_conv}.
    \item \texttt{Expand[BinomialTheoremAndDiscreteConvolutionStrictTest[m, x + y]]} verifies the Corollary~\ref{cor_bin_exp_and_macaulay_conv_strict}.
    \item \texttt{ConvPowerIdentityParametric[m, x, a]} verifies equation~\eqref{eq:parametric-identity}.
    \item \texttt{ConvPowerIdentityStrictParametric[m, x, a]} verifies equation~\eqref{eq:parametric-identity-strict}.
\end{itemize}

\subsection{Examples} \label{subsec:examples}
For example, given $m=1$ we have the following values of $\polynomialL{1}{x}{k}$
\begin{table}[H]
    \begin{tabular}{c|cccccccc}
        $x/k$ & 0 & 1  & 2  & 3  & 4  & 5  & 6  & 7 \\[3px]
        \hline
        0     & 1 &    &    &    &    &    &    &   \\
        1     & 1 & 1  &    &    &    &    &    &   \\
        2     & 1 & 7  & 1  &    &    &    &    &   \\
        3     & 1 & 13 & 13 & 1  &    &    &    &   \\
        4     & 1 & 19 & 25 & 19 & 1  &    &    &   \\
        5     & 1 & 25 & 37 & 37 & 25 & 1  &    &   \\
        6     & 1 & 31 & 49 & 55 & 49 & 31 & 1  &   \\
        7     & 1 & 37 & 61 & 73 & 73 & 61 & 37 & 1
    \end{tabular}
    \caption{Values of $\polynomialL{1}{x}{k}$.}
    \label{tab:tab_3}
\end{table}
From Table~\ref{tab:tab_3} it is seen that
\begin{equation*}
    \begin{split}
        \polynomialP{1}{0}{0} &= 0 = 0^3 \\
        \polynomialP{1}{1}{1} &= 1 = 1^3 \\
        \polynomialP{1}{2}{2} &= 1+7 = 2^3 \\
        \polynomialP{1}{3}{3} &= 1+13+13 = 3^3 \\
        \polynomialP{1}{4}{4} &= 1+19+25+19 = 4^3 \\
        \polynomialP{1}{5}{5} &= 1+25+37+37+25 = 5^3
    \end{split}
\end{equation*}
Another case, given $m=2$ we have the following values of $\polynomialL{2}{x}{k}$
\begin{table}[H]
    \begin{tabular}{c|cccccccc}
        $x/k$ & 0 & 1    & 2    & 3    & 4    & 5    & 6    & 7 \\ [3px]
        \hline
        0     & 1 &      &      &      &      &      &      &   \\
        1     & 1 & 1    &      &      &      &      &      &   \\
        2     & 1 & 31   & 1    &      &      &      &      &   \\
        3     & 1 & 121  & 121  & 1    &      &      &      &   \\
        4     & 1 & 271  & 481  & 271  & 1    &      &      &   \\
        5     & 1 & 481  & 1081 & 1081 & 481  & 1    &      &   \\
        6     & 1 & 751  & 1921 & 2431 & 1921 & 751  & 1    &   \\
        7     & 1 & 1081 & 3001 & 4321 & 4321 & 3001 & 1081 & 1
    \end{tabular}
    \caption{Values of $\polynomialL{2}{x}{k}$.}
    \label{tab:tab_4}
\end{table}
Again, an odd-power identity~\ref{lemma_polynomial_p_and_odd_power} holds
\begin{equation*}
    \begin{split}
        \polynomialP{2}{0}{0} &= 0 = 0^5 \\
        \polynomialP{2}{1}{1} &= 1 = 1^5 \\
        \polynomialP{2}{2}{2} &= 1+31 = 2^5 \\
        \polynomialP{2}{3}{3} &= 1+121+121 = 3^5 \\
        \polynomialP{2}{4}{4} &= 1+271+481+271 = 4^5 \\
        \polynomialP{2}{5}{5} &= 1+481+1081+1081+481 = 5^5
    \end{split}
\end{equation*}
Tables ~\ref{tab:tab_3}, ~\ref{tab:tab_4} are entries \href{https://oeis.org/A287326}{\texttt{A287326}},
\href{https://oeis.org/A300656}{\texttt{A300656}} in~\cite{Sloane_theencyclopedia}.


    \section{Acknowledgements}
    \label{sec:acknowledgements}
    I'd like to thank to Dr. Max Alekseyev for sufficient help in the derivation of the real coefficients $\coeffA{m}{r}$.
Also, I'd like to thank to OEIS editors Michel Marcus, Peter Luschny, Jon E. Schoenfield and others
for their useful volunteer work and for useful comments during the work on OEIS sequences related to this manuscript.



    \section{Conclusion}
    \label{sec:conclusion}
    In this manuscript we have shown that Binomial theorem is partial case of polynomial $\polynomialP{m}{b}{x}$.
Furthermore, by means of $\polynomialP{m}{b}{x}$ it is shown a relation between Binomial theorem
and discrete convolution of polynomials.


    \bibliographystyle{unsrt}
    \bibliography{OnTheBinomialTheoremAndDiscreteConvolutionReferences}
    \noindent \textbf{Version:} \texttt{Local-0.1.0}
\end{document}
