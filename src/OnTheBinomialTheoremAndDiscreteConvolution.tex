\documentclass[12pt,letterpaper,oneside,reqno]{amsart}
\usepackage{amsfonts}
\usepackage{amsmath}
\usepackage{amssymb}
\usepackage{amsthm}
\usepackage{float}
\usepackage{mathrsfs}
\usepackage[font=small,labelfont=bf]{caption}
\usepackage[left=1in,right=1in,bottom=1in,top=1in]{geometry}
\usepackage[pdfpagelabels,hyperindex,colorlinks=true,linkcolor=blue,urlcolor=magenta,citecolor=green]{hyperref}
\usepackage{graphicx}
\linespread{1.7}
\emergencystretch=1em
\usepackage{array}
\usepackage{etoolbox}
\apptocmd{\sloppy}{\hbadness 10000\relax}{}{}
\raggedbottom

\newcommand \convPower [3]{(#1 \sp{#2} \ast #1 \sp{#2}) [#3]}
\newcommand \bernoulli [2][B] {{#1}\sb{#2}}
\newcommand \multifoldSum [2][x]{{#1}\sb{1} + {#1}\sb{2} + \cdots + {#1}\sb{#2}}
\newcommand \iversonBracket [1][s]{[#1 \; \mathrm{is} \; \mathrm{even}]}
\newcommand \floor [1]{\lfloor #1 \rfloor}
\newcommand \coeffA [3][A] {{\mathbf{#1}} \sb{#2,#3}}
\newcommand \coeffH [4][H] {{\mathbf{#1}} \sb{#2,#3} (#4)}
\newcommand \polynomialX [4][X] {{\mathbf{#1}}\sb{#2,#3} (#4)}
\newcommand \polynomialP [4][P]{{\mathbf{#1}}\sp{#2} \sb{#3}(#4)}
\newcommand \polynomialL [4][L]{{\mathbf{#1}}\sb{#2}(#3,#4)}
\newcommand \polynomialT [3][T]{{\mathbf{#1}}\sp{#2} (#3)}

% free foot note
\let\svthefootnote\thefootnote
\newcommand\freefootnote[1]{%
    \let\thefootnote\relax%
    \footnotetext{#1}%
    \let\thefootnote\svthefootnote%
}

\newtheorem{thm}{Theorem}[section]
\newtheorem{cor}[thm]{Corollary}
\newtheorem{prop}[thm]{Proposition}
\newtheorem{lem}[thm]{Lemma}
\newtheorem{ppty}[thm]{Property}
\newtheorem{quest}[thm]{Question}
\numberwithin{equation}{section}

%--------Meta Data: Fill in your info------
\title[On the link between binomial theorem and discrete convolution]
{On the link between binomial theorem and discrete convolution}
\author[Petro Kolosov]{Petro Kolosov}
\email{kolosovp94@gmail.com}
\keywords{Binomial theorem, Faulhaber's formula, Discrete convolution, Polynomial identities, Power sums, Multinomial theorem}
\urladdr{https://kolosovpetro.github.io}
\subjclass[2010]{44A35, 11C08}
\date{\today}
\hypersetup{
    pdftitle={On the link between binomial theorem and discrete convolution},
    pdfsubject={
        Binomial theorem,
        Discrete convolution,
        Power function,
        Polynomials,
        Convolution,
        Multinomial theorem,
        Binomial coefficients,
        Bernoulli numbers,
        Pascal triangle,
        Faulhaber's formula,
        Power sums,
        Worpitzky identity
    },
    pdfauthor={Petro Kolosov},
    pdfkeywords={
        Binomial theorem,
        Discrete convolution,
        Power function,
        Polynomials,
        Convolution,
        Multinomial theorem,
        Binomial coefficients,
        Bernoulli number,
        Pascal's triangle,
        Faulhaber's formula,
        Power sums,
        Worpitzky identity
    }
}
\begin{document}
    \begin{abstract}
        \input{sections/01-abstract}
    \end{abstract}

    \maketitle

    \tableofcontents

    \freefootnote{Sources: \url{https://github.com/kolosovpetro/OnTheBinomialTheoremAndDiscreteConvolution}}


    \section{Definitions, notations and conventions} \label{sec:definitions-notations-and-conventions}
    We now set the following notation, which remains fixed for the remainder of this manuscript
\begin{itemize}
    \setlength\itemsep{1.6em}
    \item $\coeffA{m}{r}$ is a real coefficient defined recursively
    \begin{equation}
        \label{eq:def_coeff_a}
        \coeffA{m}{r} =
        \begin{cases}
        (2r+1)
            \binom{2r}{r} & \text{if } r=m \\
            (2r+1) \binom{2r}{r} \sum_{d=2r+1}^{m} \coeffA{m}{d} \binom{d}{2r+1} \frac{(-1)^{d-1}}{d-r}
            \bernoulli{2d-2r} & \text{if } 0 \leq r<m \\
            0 & \text{if } r<0 \text{ or } r>m
        \end{cases}
    \end{equation}
    where $m$ is non-negative integer and $\bernoulli{t}$ are Bernoulli numbers~\cite{WeissteinBernoulli}.
    It is assumed that $\bernoulli{1}=\frac{1}{2}$.

    \item $\polynomialP{m}{b}{x}$ is a $2m+1$-degree polynomial in $b,x\in\mathbb{R}$
    \begin{equation}
        \label{eq:def_polynomial_p}
        \polynomialP{m}{b}{x} = \sum_{k=0}^{b-1} \sum_{r=0}^{m} \coeffA{m}{r} k^r(x-k)^r
    \end{equation}

    \item $\coeffH{m}{t}{b}$ is a polynomial defined as
    \begin{equation}
        \label{eq:def_coeff_h}
        \coeffH{m}{t}{b}
        = \sum_{j=t}^{m} \binom{j}{t} \coeffA{m}{j} \frac{(-1)^j}{2j-t+1} \binom{2j-t+1}{b} \bernoulli{2j-t+1-b}
    \end{equation}
    integers $m,t,b$.

    \item $\polynomialX{m}{t}{j}$ is polynomial of degree $2m+1-t$ in $j\in\mathbb{R}$
    \begin{equation}
        \label{eq:def_coeff_x}
        \polynomialX{m}{t}{j} = (-1)^m \sum_{k=1}^{2m+1-t} \coeffH{m}{t}{k} \cdot j^k
    \end{equation}
    integers $m,t$.

    \item $\polynomialL{m}{x}{k}$ is $2m$ degree polynomial in $x,k\in\mathbb{R}$
    \begin{equation}
        \label{eq:def_polynomial_l}
        \polynomialL{m}{x}{k} = \sum_{r=0}^{m} \coeffA{m}{r} k^r(x-k)^r
    \end{equation}

    \item $(f\ast f)[n]$ is discrete convolution~\cite{damelin_discrete_convolution} of function $f$ defined over set of integers $\mathbb{Z}$
    \begin{align*}
    (f\ast f)[n]
        = \sum_{k} f(k) f(n-k)
    \end{align*}
    and its partial case for polynomials $n^j, \; n\geq a \in \mathbb{R}$
    \begin{align*}
        \convPower{n}{j}{x} =\sum_{k} k^j (x-k)^j [k\geq a][x-k\geq a] =\sum_{k=a}^{x-a} k^j (x-k)^j
    \end{align*}
\end{itemize}

    \clearpage


    \section{Introduction and main results} \label{sec:introduction}
    The polynomial $\polynomialP{m}{b}{x}$ is a $2m+1$-degree polynomial in $x,b\in\mathbb{R}$ defined as
\begin{align*}
    \polynomialP{m}{b}{x} = \sum_{k=0}^{b-1} \sum_{r=0}^{m} \coeffA{m}{r} k^r(x-k)^r
\end{align*}
where $\coeffA{m}{r}$ is a real coefficient.
By means of Lemma~\ref{lemma_polynomial_p_and_odd_power},
the polynomial $\polynomialP{m}{b}{x}$ has the following relation with Binomial theorem~\cite{AbraSteg72}
\begin{align*}
    \polynomialP{m}{x+y}{x+y} = \sum_{r=0}^{2m+1} \binom{2m+1}{r} x^{2m+1-r} y^r
\end{align*}
On the other hand, polynomial $\polynomialP{m}{b}{x}$ might be expressed in terms of discrete convolution
of polynomial $n^j$.
For every $n\geq 0$
\begin{align*}
    \polynomialP{m}{x+1}{x} = \sum_{r=0}^{m} \coeffA{m}{r} \convPower{n}{r}{x}
\end{align*}
It is important to notice that  $n^r$ of discrete convolution $\convPower{n}{r}{x}$ evaluated at $x$
is implicit piecewise-defined polynomial such as
\begin{equation*}
    n^{r} =
    \begin{cases}
        \underbrace{n \cdot n \cdots n}_{\mathrm{r \; times}}, & \mbox{if } n \geq 0 \\
        0, & \mbox{otherwise}
    \end{cases}
\end{equation*}
Therefore, it is easy to notice the following identities in terms of Binomial theorem and discrete convolution,
see Corollaries~\ref{cor_bin_exp_and_macaulay_conv} and~\ref{cor_bin_exp_and_macaulay_conv_strict}.
For every $n \geq 0$
\begin{equation*}
    \sum_{r=0}^{m} \coeffA{m}{r} \convPower{n}{r}{x+y}
    = 1 + \sum_{r=0}^{2m+1} \binom{2m+1}{r} x^{2m+1-r} y^r
\end{equation*}
For every $n > 0$
\begin{equation*}
    \sum_{r=0}^{m} \coeffA{m}{r} \convPower{n}{r}{x+y}
    = -1 + \sum_{r=0}^{2m+1} \binom{2m+1}{r} x^{2m+1-r} y^r
\end{equation*}
Additionally, the following generalizations for the multinomial case are discussed in
corollaries~\ref{cor_mult_exp_and_macaulay_conv} and ~\ref{cor_mult_exp_and_macaulay_conv_strict}.
For every $n \geq 0$
\begin{align*}
    \sum_{r=0}^{m} \coeffA{m}{r} \convPower{n}{r}{\multifoldSum{t}} =
    1 + \sum_{\multifoldSum[k]{t}=2m+1} \binom{2m+1}{k_1, k_2,\ldots, k_t} \prod_{\ell=1}^{t} x_\ell^{k_\ell}
\end{align*}
For every $n>0$
\begin{align*}
    \sum_{r=0}^{m} \coeffA{m}{r} \convPower{n}{r}{\multifoldSum{t}} =
    -1 + \sum_{\multifoldSum[k]{t}=2m+1} \binom{2m+1}{k_1, k_2,\ldots, k_t} \prod_{\ell=1}^{t} x_\ell^{k_\ell}
\end{align*}
A few polynomial identities are straightforward as well by means of
Theorems~\ref{thm_odd_power_by_macaulays_convolution},~\ref{thm_odd_power_by_macaulays_convolution_strict}.
Precisely, by the theorem~\ref{thm_odd_power_by_macaulays_convolution} we have an odd-power identity as follows
\begin{equation*}
    x^{2m+1} = \sum_{r=0}^{m} \coeffA{m}{r} \sum_{k=0}^{x-1} k^r (x-k)^r
\end{equation*}
From the other side, the theorem~\eqref{thm_odd_power_by_macaulays_convolution_strict} provides an odd-power
polynomial identity as follows
\begin{equation*}
    x^{2m+1} = \sum_{r=0}^{m} \coeffA{m}{r} \sum_{k=1}^{x} k^r (x-k)^r
\end{equation*}
For example,
\begin{align*}
    x^3 &= \sum_{k=1}^{x} 6k (x-k) + 1 \\
    x^5 &= \sum_{k=1}^{x} 30k^2 (x-k)^2 + 1 \\
    x^7 &= \sum_{k=1}^{x} 140 k^3 (x-k)^3 - 14k(x-k) + 1 \\
    x^9 &= \sum_{k=1}^{x} 630 k^4(x-k)^4 - 120k(x-k) + 1 \\
    x^{11} &= \sum_{k=1}^{x} 2772 k^5 (x-k)^5 + 660 k^2(x-k)^2 - 1386k(x-k) + 1 \\
    x^{13} &= \sum_{k=1}^{x} 51480 k^7 (x-k)^7 - 60060 k^3 (x-k)^3 + 491400 k^2 (x-k)^{2} - 450054 k (x-k) + 1 \\
\end{align*}



    \section{Polynomial \texorpdfstring{$\polynomialP{m}{b}{x}$}{P[m,b,x]} and its properties}
    \label{sec:polynomial-p-and-its-properties}
    \input{sections/04-polynomial-p-and-its-properties}


    \section{Relation between the polynomial \texorpdfstring{$\polynomialP{m}{b}{x}$}{P[m,b,x]} and Binomial theorem}
    \label{sec:odd-binomial-expansion-as-partial-case-of-polynomial-p}
    \begin{lem}
    \label{lemma_polynomial_p_and_odd_power}
    For every $m\in\mathbb{N}, \; x,y\in\mathbb{R}$
    \begin{equation*}
        \polynomialP{m}{x+y}{x+y} = \sum_{r=0}^{2m+1} \binom{2m+1}{r} x^{2m+1-r} y^r
    \end{equation*}
\end{lem}
By Lemma~\ref{lemma_polynomial_p_and_odd_power} and equation~\eqref{eq:p_all_forms} the following
polynomial identities straightforward
\begin{equation*}
    x^{2m+1}
    =\sum_{r=0}^{m} \sum_{\ell=1}^{2m-r+1} (-1)^{2m-r} \coeffH{m}{r}{\ell} \cdot x^{\ell+r}
    =\sum_{r=0}^{m} (-1)^{m-r} \polynomialX{m}{r}{x} \cdot x^r
\end{equation*}
For instance,
\begin{equation*}
    \polynomialP{2}{x+y}{x+y} = (x + y) (x^4 + 4 x^3 y + 6 x^2 y^2 + 4 x y^3 + y^4).
\end{equation*}
In addition, the following identities hold
\begin{equation*}
    \begin{split}
    (x+y)
        ^{2m+1}
        &=\sum_{r=0}^{m} \sum_{\ell=1}^{2m-r+1} (-1)^{2m-r} \coeffH{m}{r}{\ell} \cdot (x+y)^{\ell+r} \\
        &=\sum_{r=0}^{m} (-1)^{m-r} \polynomialX{m}{r}{x+y} \cdot (x+y)^r
    \end{split}
\end{equation*}
Obviously, Multinomial expansion of $t$-fold sum $(\multifoldSum{t})^{2m+1}$ can be reached
by $\polynomialP{m}{b}{\multifoldSum{t}}$ as well
\begin{cor}
    For all $x_1,x_2,\ldots, x_t\in\mathbb{R}, \; m \in \mathbb{N}$
    \begin{equation*}
        \polynomialP{m}{\multifoldSum{t}}{\multifoldSum{t}}
        =
        \sum_{\multifoldSum[k]{t}=2m+1} \binom{2m+1}{k_1, k_2,\ldots, k_t} \prod_{s=1}^{t} x_t^{k_s}
    \end{equation*}
\end{cor}
Moreover, the following multinomial identities hold
\begin{equation*}
    \begin{split}
    (\multifoldSum{t})
        ^{2m+1}
        &=\sum_{r=0}^{m} \sum_{\ell=1}^{2m-r+1} (-1)^{2m-r} \coeffH{m}{r}{\ell} \cdot (\multifoldSum{t})^{\ell+r} \\
        &=\sum_{r=0}^{m} (-1)^{m-r} \polynomialX{m}{r}{\multifoldSum{t}} \cdot (\multifoldSum{t})^r
    \end{split}
\end{equation*}



    \section{Polynomial \texorpdfstring{$\polynomialP{m}{b}{x}$}{P[m,b,x]} in terms of Discrete convolution}
    \label{sec:relation-between-p-and-convolution-of-polynomials}
    In this section we discuss the relation between $\polynomialP{m}{b}{x}$ and discrete convolution of
polynomials.
To show that $\polynomialP{m}{b}{x}$ involves the discrete convolution of polynomial $n^r$
recall the definition of the polynomial $\polynomialP{m}{b}{x}$
\begin{equation*}
    \polynomialP{m}{b}{x} = \sum_{k=0}^{b-1} \sum_{r=0}^{m} \coeffA{m}{r} k^r (x-k)^r
    = \sum_{r=0}^{m} \coeffA{m}{r} \sum_{k=0}^{b-1} k^r (x-k)^r
\end{equation*}
A discrete convolution of defined over set of integers $\mathbb{Z}$ function $f$ is
\begin{equation*}
(f \ast f)[n]
    = \sum_{k} f(k) f(n-k)
\end{equation*}
General formula of discrete convolution for polynomials $f(n) = n^j, \; n\geq a \in \mathbb{R}$
can be derived immediately
\begin{equation*}
    \begin{split}
        \convPower{n}{j}{x}
        &=\sum_{k} k^j (x-k)^j [k\geq a][x-k\geq a] \\
        &=\sum_{k} k^j (x-k)^j [k\geq a][k\leq x-a] \\
        &=\sum_{k} k^j (x-k)^j [a \leq k \leq x-a] \\
        &=\sum_{k=a}^{x-a} k^j (x-k)^j
    \end{split}
\end{equation*}
where $[a \leq k \leq x-a]$ is Iverson's bracket~\cite{iverson_apl, knuth_two_notes_on_notation}.
\begin{lem}
    \label{lemma_disc_conv_identity}
    For every $n\in\mathbb{N}, \; x\in\mathbb{R}$ and $n\geq 0$
    \[
        \convPower{n}{r}{x} = \sum_{k=0}^{x} k^r (x-k)^r
    \]
\end{lem}
It is of first importance to keep in mind that  $n^r$ of discrete convolution $\convPower{n}{r}{x}$ evaluated at $x$
is an implicit piecewise-defined polynomial such as
\begin{equation*}
    n^{r} =
    \begin{cases}
        \underbrace{n \cdot n \cdots n}_{\mathrm{r \; times}}, & \mbox{if } n \geq 0 \\
        0, & \mbox{otherwise}
    \end{cases}
\end{equation*}
Thus, the corollary follows
\begin{cor}
    \label{cor_polynomial_p_and_macaulay_convolution}
    By Lemma~\ref{lemma_disc_conv_identity} the polynomial $\polynomialP{m}{b}{n}$ might be expressed in terms
    of discrete convolution as follows, for every $n\geq 0$
    \begin{align*}
        \polynomialP{m}{x+1}{x} = \sum_{r=0}^{m} \coeffA{m}{r} \convPower{n}{r}{x}
    \end{align*}
\end{cor}
Therefore, another polynomial identity follows
\begin{thm}
    \label{thm_odd_power_by_macaulays_convolution}
    By Lemma~\ref{lemma_polynomial_p_and_odd_power}, Corollary~\ref{cor_polynomial_p_and_macaulay_convolution}
    and property~\ref{prop_p_identity}, for every $m\in\mathbb{N}, \; x\in\mathbb{R}$ and $n\geq 0$
    \begin{align*}
        1 + x^{2m+1} = \sum_{r=0}^{m} \coeffA{m}{r} \convPower{n}{r}{x}
                     = \sum_{r=0}^{m} \coeffA{m}{r} \sum_{k=0}^{x} k^r (x-k)^r
    \end{align*}
\end{thm}
Now we notice the following identity in terms of polynomial $\polynomialP{m}{b}{x}$ and
discrete convolution $\convPower{n}{j}{x}$
\begin{prop}
    \label{prop_polynomial_p_and_macaulay_convolution_strict}
    For every $m \in \mathbb{N}, \; x\in\mathbb{R}$ and $n \geq 1$
    \begin{equation*}
        \begin{split}
            \polynomialP{m}{x}{x}
            &=\sum_{r=0}^{m} \coeffA{m}{r} \left(0^r x^r + \sum_{k=1}^{x-1} k^r (x-k)^r \right) \\
            &=\sum_{r=0}^{m} \coeffA{m}{r} 0^r x^r + \sum_{r=0}^{m} \coeffA{m}{r} \convPower{n}{r}{x} \\
            &=1 + \sum_{r=0}^{m} \coeffA{m}{r} \convPower{n}{r}{x}
        \end{split}
    \end{equation*}
\end{prop}
Since that for all $r$ in $\coeffA{m}{r} 0^r x^r$ we have
\begin{equation*}
    \coeffA{m}{r} 0^r x^r =
    \begin{cases}
        1, & \mbox{if } r=0 \\
        0, & \mbox{if } r>0
    \end{cases}
\end{equation*}
Above is true because $\coeffA{m}{0}=1$ for every $m\in\mathbb{N}$, and $x^0 = 1$
for every $x$, see~\cite{graham1994concrete}.
Hence, the following identity between $\polynomialP{m}{b}{x}$ and
discrete convolution $\convPower{n}{j}{x}$ holds
\begin{thm}
    \label{thm_odd_power_by_macaulays_convolution_strict}
    By Lemma~\ref{lemma_polynomial_p_and_odd_power} and
    Proposition~\ref{prop_polynomial_p_and_macaulay_convolution_strict},
    for every $m\in\mathbb{N}, \; x\in\mathbb{R}$ and $n > 0$
    \begin{equation*}
        -1 + x^{2m+1} = \sum_{r=0}^{m} \coeffA{m}{r} \convPower{n}{r}{x}
                      = \sum_{r=0}^{m} \coeffA{m}{r} \sum_{k=1}^{x-1} k^r (x-k)^r
    \end{equation*}
\end{thm}
\begin{cor}
    \label{cor_sum_of_coeffs_a}
    By Theorem~\ref{thm_odd_power_by_macaulays_convolution_strict}, for all $m\in\mathbb{N}$
    \begin{equation*}
        \sum_{r=0}^{m} \coeffA{m}{r} = 2^{2m+1} - 1
    \end{equation*}
\end{cor}
Corollary~\ref{cor_sum_of_coeffs_a} holds since that convolution $\convPower{n}{j}{x}=1, \; n > 0$
for each $r$ and $x=2$.



    \section{Relation between Binomial theorem and Discrete convolution}
    \label{sec:relation-between-binomial-theorem-and-discrete-convolution}
    \begin{cor}
    \label{cor_bin_exp_and_macaulay_conv}
    (Generalization of Theorem~\ref{thm_odd_power_by_macaulays_convolution} for Binomials.)
    For every $m\in\mathbb{N}, \; x,y\in\mathbb{R}$ and $n\geq 0$
    \begin{equation*}
        \sum_{r=0}^{m} \coeffA{m}{r} \convPower{n}{r}{x+y}
        =
        1 + \sum_{r=0}^{2m+1} \binom{2m+1}{r} x^{2m+1-r} y^r
    \end{equation*}
\end{cor}
For example, given $m=0,1,2$ the Corollary~\ref{cor_bin_exp_and_macaulay_conv} yields
\begin{align*}
    \sum_{r=0}^{0} \coeffA{0}{r} \convPower{n}{r}{x+y}
    &= 1 + x + y \\
    \sum_{r=0}^{1} \coeffA{1}{r} \convPower{n}{r}{x+y}
    &= 1 + x + y - (x + y) (1 + x + y) (1 - 3 x - 3 y + 2 (x + y)) \\
    &= 1 + x^3 + 3 x^2 y + 3 x y^2 + y^3\\
    \sum_{r=0}^{2} \coeffA{2}{r} \convPower{n}{r}{x+y}
    &=1 + x + y + (x + y) (1 + x + y) \left(-1 + x + 5 x^2 + y + 10 x y + 5 y^2\right. \\
    &-15 x (x + y) + 10 x^2 (x + y) - 15 y (x + y) + 20 x y (x + y) \\
    &+ 10 y^2 (x + y) +9 (x + y)^2 - 15 x (x + y)^2 \\
    &\left.-15 y (x + y)^{2} + 6 {(x + y)}^{3}\right) \\
    &=x^5 + 5 x^4 y + 10 x^3 y^2 + 10 x^2 y^3 + 5 x y^4 + y^5 + 1
\end{align*}
Above example could be verified using using the commands defined in Mathematica package at~\cite{github_source_files}
\begin{itemize}
    \item \texttt{BinomialTheoremAndDiscreteConvolutionTest[0, x + y]}
    \item \texttt{BinomialTheoremAndDiscreteConvolutionTest[1, x + y]}
    \item \texttt{Expand[BinomialTheoremAndDiscreteConvolutionTest[1, x + y]]}
    \item \texttt{BinomialTheoremAndDiscreteConvolutionTest[2, x + y]}
    \item \texttt{Expand[BinomialTheoremAndDiscreteConvolutionTest[2, x + y]]}
\end{itemize}
\begin{cor}
    \label{cor_bin_exp_and_macaulay_conv_strict}
    (Generalization of Theorem~\ref{thm_odd_power_by_macaulays_convolution_strict} for Binomials.)
    For every $m\in\mathbb{N}, \; x,y\in\mathbb{R}$ and $n > 0$
    \begin{equation*}
        \sum_{r=0}^{m} \coeffA{m}{r} \convPower{n}{r}{x+y}
        =
        -1 + \sum_{r=0}^{2m+1} \binom{2m+1}{r} x^{2m+1-r} y^r
    \end{equation*}
\end{cor}
For example, given $m=0,1$ the Corollary~\ref{cor_bin_exp_and_macaulay_conv_strict} gives
\begin{equation*}
    \begin{split}
        \sum_{r=0}^{0} \coeffA{0}{r} \convPower{n}{r}{x+y}
        &= x + y - 1 \\
        \sum_{r=0}^{1} \coeffA{1}{r} \convPower{n}{r}{x+y}
        &= -1 + x + y - (-1 + x + y) (x + y) (-1 - 3 x - 3 y + 2 (x + y)) \\
        &= x^3 + 3 x^2 y + 3 x y^2 + y^3 - 1
    \end{split}
\end{equation*}
Above example could be verified using using the commands defined in Mathematica package at~\cite{github_source_files}
\begin{itemize}
    \item \texttt{BinomialTheoremAndDiscreteConvolutionStrictTest[0, x + y]}
    \item \texttt{BinomialTheoremAndDiscreteConvolutionStrictTest[1, x + y]}
    \item \texttt{Expand[BinomialTheoremAndDiscreteConvolutionStrictTest[1, x + y]]}
\end{itemize}
From the other prospective, the following binomial holds.
For every $n \geq 0$
\begin{equation}
    \label{eq:parametric-identity}
    \begin{split}
        (x-2a)^{2m+1} + 1 &= \sum_{r=0}^{m} \coeffA{m}{r} ((t-k)^r \ast (t-k)^r)[x] \\
                          &= \sum_{r=0}^{m} \coeffA{m}{r} \sum_{k=a}^{x-a} (k-a)^r (x-k-a)^r
    \end{split}
\end{equation}
Similarly, the following binomial holds.
For every $n > 0$
\begin{equation}
    \label{eq:parametric-identity-strict}
    \begin{split}
        (x-2a)^{2m+1} - 1 &= \sum_{r=0}^{m} \coeffA{m}{r} ((t-k)^r \ast (t-k)^r)[x] \\
                          &= \sum_{r=0}^{m} \coeffA{m}{r} \sum_{k=a+1}^{x-a-1} (k-a)^r (x-k-a)^r
    \end{split}
\end{equation}
To validate equations~\eqref{eq:parametric-identity} and~\eqref{eq:parametric-identity-strict}
use the following commands
\begin{itemize}
    \item \texttt{ConvolutionOfBinomial[10, 2, 1]} verifies an equation~\eqref{eq:parametric-identity}.
    \item \texttt{ConvolutionOfBinomial1[10, 2, 1]} verifies an equation~\eqref{eq:parametric-identity-strict}.
\end{itemize}

\subsection{Generalization for Multinomials} \label{subsec:generalization-for-multinomials}
In this subsection we generalize
Theorems~\eqref{thm_odd_power_by_macaulays_convolution} and~\eqref{thm_odd_power_by_macaulays_convolution_strict}
for multinomial cases.
\begin{cor}
    \label{cor_mult_exp_and_macaulay_conv}
    (Generalization of Theorem~\ref{thm_odd_power_by_macaulays_convolution} for Multinomials.)
    For every $x_1, x_2, \ldots, x_t\in\mathbb{R}, \; m\in\mathbb{N}, \; n \geq 1$
    \[
        \sum_{r=0}^{m} \coeffA{m}{r} \convPower{n}{r}{\multifoldSum{t}} =
        1 + \sum_{\multifoldSum[k]{t}=2m+1} \binom{2m+1}{k_1, k_2,\ldots, k_t} \prod_{\ell=1}^{t} x_\ell^{k_\ell}
    \]
\end{cor}
For instance, given $m=1$ the Corollary~\ref{cor_mult_exp_and_macaulay_conv} gives
\begin{equation*}
    \begin{split}
        &\sum_{r=0}^{1} \coeffA{1}{r} \convPower{n}{r}{x+y+z} \\
        &=1 + x + y + z - (x + y + z) (1 + x + y + z) (1 - 3 x - 3 y - 3 z + 2 (x + y + z)) \\
        &=1 + x^3 + 3 x^2 y + 3 x y^2 + y^3 + 3 x^2 z + 6 x y z + 3 y^2 z + 3 x z^2 + 3 y z^2 + z^3.
    \end{split}
\end{equation*}
Above example could be verified using using the commands defined in Mathematica package at~\cite{github_source_files}
\begin{itemize}
    \item \texttt{BinomialTheoremAndDiscreteConvolutionTest[1, x + y + z]}
    \item \texttt{Expand[BinomialTheoremAndDiscreteConvolutionTest[1, x + y + z]]}
\end{itemize}
\begin{cor}
    \label{cor_mult_exp_and_macaulay_conv_strict}
    (Generalization of Theorem~\ref{thm_odd_power_by_macaulays_convolution_strict} for Multinomials.)
    For each $\multifoldSum{t} \geq 1, \; x_1,x_2,\ldots,x_t\in\mathbb{R}, \; m\in\mathbb{N}, \; n\geq1$
    \[
        \sum_{r=0}^{m} \coeffA{m}{r} \convPower{n}{r}{\multifoldSum{t}} =
        -1 + \sum_{\multifoldSum[k]{t}=2m+1} \binom{2m+1}{k_1, k_2,\ldots, k_t} \prod_{\ell=1}^{t} x_\ell^{k_\ell}
    \]
\end{cor}
For example, given $m=1$ the Corollary~\ref{cor_mult_exp_and_macaulay_conv_strict} gives
\begin{equation*}
    \begin{split}
        &\sum_{r=0}^{1} \coeffA{1}{r} \convPower{n}{r}{x+y+z} \\
        &=-1 + x + y + z - (-1 + x + y + z) (x + y + z) (-1 - 3 x - 3 y - 3 z + 2 (x + y + z)) \\
        &=-1 + x^3 + 3 x^2 y + 3 x y^2 + y^3 + 3 x^2 z + 6 x y z + 3 y^2 z + 3 x z^2 + 3 y z^2 + z^3.
    \end{split}
\end{equation*}
Above example could be verified using using the commands defined in Mathematica package at~\cite{github_source_files}
\begin{itemize}
    \item \texttt{BinomialTheoremAndDiscreteConvolutionStrictTest[1, x + y + z]}
    \item \texttt{Expand[BinomialTheoremAndDiscreteConvolutionStrictTest[1, x + y + z]]}
\end{itemize}



    \section{Derivation of the coefficient \texorpdfstring{$\coeffA{m}{r}$}{A[m,r]}}
    \label{sec:derivation-of-coefficients-a}
    \input{sections/08-derivation-of-the-coefficient-a}


    \section{Conclusion}
    \label{sec:conclusion}
    In this manuscript, we introduced the polynomial $\mathbf{P}^{m}_{b}(x)$ and examined its properties.
We established a polynomial identity for odd-powers that demonstrates the connection between Binomial theorem
and discrete convolution of odd-powered polynomials.
This relationship was extended to the multinomial case.
All results were verified using Mathematica programs.



    \section{Acknowledgements}
    \label{sec:acknowledgements}
    I'd like to thank to Dr. Max Alekseyev for sufficient help in the derivation of the real coefficients $\coeffA{m}{r}$.
Also, I'd like to thank to OEIS editors Michel Marcus, Peter Luschny, Jon E. Schoenfield and others
for their useful volunteer work and for useful comments during the work on OEIS sequences related to this manuscript.



    \bibliographystyle{unsrt}
    \bibliography{OnTheBinomialTheoremAndDiscreteConvolutionReferences}
    \noindent \textbf{Version:} \input{sections/version}

    \clearpage


    \section{Addendum 1: Verification of the results}
    \label{sec:verification-of-the-results-and-examples}
    To fulfill our study we provide an opportunity to verify its results by means of Wolfram Mathematica language.

\subsection{Mathematica commands} \label{subsec:mathematica-commands}
Proceeding to the repository~\cite{github_source_files} reader is able to find there a folder named \texttt{mathematica}
that contains the files
\begin{itemize}
    \item \texttt{OnTheBinomialTheoremAndDiscreteConvolution.m} is a package file with definitions
    \item \texttt{OnTheBinomialTheoremAndDiscreteConvolution.nb} is a notebook file with examples.
\end{itemize}
The following commands may be used to reproduce the results of this manuscript:
\begin{itemize}
    \item \texttt{A[m, r]} returns the real coefficient $\coeffA{m}{r}$ defined by~\eqref{eq:def_coeff_a}.
    \item \texttt{PrintTriangleOfA[rows]} prints the table of coefficients $\coeffA{m}{r}$. \\
    Command \texttt{PrintTriangleOfA[7]} reproduces the table (\ref{tab:table_of_coefficients_a}).
    \item \texttt{PolynomialL[m, n, k]} returns the polynomial $\polynomialL{m}{n}{k}$ defined by~\eqref{eq:def_polynomial_l}.
    \item \texttt{PolynomialP[m, x, b]} returns the polynomial $\polynomialP{m}{b}{x}$ defined by~\eqref{eq:def_polynomial_p}.
    \item \texttt{Expand[PolynomialP[m, x + y, x + y]]} verifies the Lemma~\ref{lemma_polynomial_p_and_odd_power}.
    \item \texttt{PolynomialH[m, t, j]} returns the polynomial $\coeffH{m}{t}{j}$ defined by~\eqref{eq:def_coeff_h}.
    \item \texttt{PolynomialX[m, t, k]} returns the polynomial $\polynomialX{m}{t}{k}$ defined by~\eqref{eq:def_coeff_x}.
    \item \texttt{Expand[BinomialTheoremAndDiscreteConvolutionTest[m, x + y]]} verifies the Corollary~\ref{cor_bin_exp_and_macaulay_conv}.
    \item \texttt{Expand[BinomialTheoremAndDiscreteConvolutionStrictTest[m, x + y]]} verifies the Corollary~\ref{cor_bin_exp_and_macaulay_conv_strict}.
    \item \texttt{DiscreteConvolutionPowerIdentityParametricTest[m, x, a]} verifies an equation~\eqref{eq:parametric-identity}.
    Usage \texttt{Column[Table[DiscreteConvolutionPowerIdentityParametricTest[1, x, 1], {x, 3, 20}], Left]}.
    \item \texttt{DiscreteConvolutionPowerIdentityStrictParametricTest[m, x, a]} verifies an equation~\eqref{eq:parametric-identity-strict}.
    Usage \texttt{Column[Table[DiscreteConvolutionPowerIdentityStrictParametricTest[1, x, 1], {x, 3, 20}], Left]}.
    \item \texttt{Expand[PolynomialIdentityOfP[1, n, b]]} validates an identity
    \[\polynomialP{m}{b}{x} = \sum_{r=0}^{m} \coeffA{m}{r} \sum_{j=0}^{r} (-1)^{j} x^{r-j} \binom{r}{j} \sum_{k=0}^{b-1} k^{r+j}\]
    \item \texttt{PolynomialIdentityInvolvingX[m, x, b]} validates an identity~\eqref{eq:p_all_forms}
    \[\polynomialP{m}{b}{x} = \sum_{r=0}^{m} (-1)^{m-r} \polynomialX{m}{r}{b} \cdot x^r\]
    \item \texttt{PolynomialIdentityInvolvingH[m, n, b]} validates an identity~\eqref{eq:p_all_forms}.
    \[\polynomialP{m}{b}{x} =\sum_{r=0}^{m} \sum_{\ell=1}^{2m-r+1} (-1)^{2m-r} \coeffH{m}{r}{\ell} \cdot b^\ell \cdot x^r\]
\end{itemize}

\subsection{Examples} \label{subsec:examples}
For example, given $m=1$ we have the following values of $\polynomialL{1}{x}{k}$
\begin{table}[H]
    \setlength\extrarowheight{-6pt}
    \begin{tabular}{c|cccccccc}
        $x/k$ & 0 & 1  & 2  & 3  & 4  & 5  & 6  & 7 \\
        \hline
        0     & 1 &    &    &    &    &    &    &   \\
        1     & 1 & 1  &    &    &    &    &    &   \\
        2     & 1 & 7  & 1  &    &    &    &    &   \\
        3     & 1 & 13 & 13 & 1  &    &    &    &   \\
        4     & 1 & 19 & 25 & 19 & 1  &    &    &   \\
        5     & 1 & 25 & 37 & 37 & 25 & 1  &    &   \\
        6     & 1 & 31 & 49 & 55 & 49 & 31 & 1  &   \\
        7     & 1 & 37 & 61 & 73 & 73 & 61 & 37 & 1
    \end{tabular}~\caption{Values of $\polynomialL{1}{x}{k}$.
    See OEIS entry: \href{https://oeis.org/A300656}{\texttt{A300656}}, \cite{kolosov2017third}.}
    \label{tab:tab_3}
\end{table}
Table~\ref{tab:tab_3} can be reproduced using Mathematica command
\begin{center}
    \texttt{PrintTriangleOfPolynomialL[1, 7]}
\end{center}
defined in the~\cite{github_source_files}.
From Table~\ref{tab:tab_3} it is seen that
\begin{equation*}
    \begin{split}
        \polynomialP{1}{0}{0} &= 0 = 0^3 \\
        \polynomialP{1}{1}{1} &= 1 = 1^3 \\
        \polynomialP{1}{2}{2} &= 1+7 = 2^3 \\
        \polynomialP{1}{3}{3} &= 1+13+13 = 3^3 \\
        \polynomialP{1}{4}{4} &= 1+19+25+19 = 4^3 \\
        \polynomialP{1}{5}{5} &= 1+25+37+37+25 = 5^3
    \end{split}
\end{equation*}
Another case, given $m=2$ we have the following values of $\polynomialL{2}{x}{k}$
\begin{table}[H]
    \setlength\extrarowheight{-6pt}
    \begin{tabular}{c|cccccccc}
        $x/k$ & 0 & 1    & 2    & 3    & 4    & 5    & 6    & 7 \\
        \hline
        0     & 1 &      &      &      &      &      &      &   \\
        1     & 1 & 1    &      &      &      &      &      &   \\
        2     & 1 & 31   & 1    &      &      &      &      &   \\
        3     & 1 & 121  & 121  & 1    &      &      &      &   \\
        4     & 1 & 271  & 481  & 271  & 1    &      &      &   \\
        5     & 1 & 481  & 1081 & 1081 & 481  & 1    &      &   \\
        6     & 1 & 751  & 1921 & 2431 & 1921 & 751  & 1    &   \\
        7     & 1 & 1081 & 3001 & 4321 & 4321 & 3001 & 1081 & 1
    \end{tabular}
    \caption{Values of $\polynomialL{2}{x}{k}$.
    See the OEIS entry \href{https://oeis.org/A300656}{\texttt{A300656}}, ~\cite{kolosov2018fifth}.}
    \label{tab:tab_4}
\end{table}
Table~\ref{tab:tab_4} can be reproduced using Mathematica command
\begin{center}
    \texttt{PrintTriangleOfPolynomialL[2, 7]}
\end{center}
defined in the~\cite{github_source_files}.
Again, an odd-power identity~\ref{lemma_polynomial_p_and_odd_power} holds
\begin{equation*}
    \begin{split}
        \polynomialP{2}{0}{0} &= 0 = 0^5 \\
        \polynomialP{2}{1}{1} &= 1 = 1^5 \\
        \polynomialP{2}{2}{2} &= 1+31 = 2^5 \\
        \polynomialP{2}{3}{3} &= 1+121+121 = 3^5 \\
        \polynomialP{2}{4}{4} &= 1+271+481+271 = 4^5 \\
        \polynomialP{2}{5}{5} &= 1+481+1081+1081+481 = 5^5
    \end{split}
\end{equation*}

\end{document}
